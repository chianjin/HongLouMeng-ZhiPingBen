%{{第六十七回}}{第六十七回}}

\chapter{馈土物颦卿念故里\\讯家童凤姐蓄阴谋}
话说尤三姐自戕之后,尤老娘以及尤二姐、贾珍、尤氏并贾蓉、贾琏等闻之,俱各不胜悲恸伤感,自不必说,忙着人治买棺木盛殓,送往城外埋葬。却说柳湘莲见尤三姐身亡,迷性不悟,尚有痴情眷恋,被道人数句偈言打破迷关,竟自削发出家,跟随疯道人飘然而去,不知何往。后事暂且不表。

且说薛姨妈闻知湘莲已说定了尤三姐为妻,心甚喜悦,正自高高兴兴要打算替他买房屋、治器用、办妆奁,择吉日迎娶过门等事,以报他救命之恩。忽有家中小厮见薛姨妈,告知尤三姐自戕与柳湘莲出家的信息,心甚叹息。正自猜疑是为什么原故,时值宝钗从园子里过来,薛姨妈便对宝钗说道:``我的儿,你听见了没有?你珍大嫂子的妹妹尤三姐,他不是已经许定了给你哥哥的义弟柳湘莲的?这也很好。不知为什么尤三姐自刎了,柳湘莲也出了家了。真正奇怪的事,叫人意想不到!''宝钗听了,并不在意,便说道:``俗语说的好,`天有不测风云,人有旦夕祸福'。这也是他们前生命定,活该不是夫妻。妈所为的是因有救哥哥的一段好处,故谆谆感叹。如果他两人齐齐全全的,妈自然该替他料理,如今死的死了,出家的出了家了,依我说,也只好由他罢了。妈也不必为他们伤感,损了自己的身子。倒是自从哥哥起江南回来了一二十日,贩了来的货物,想来也该发完了,那同伴去的伙计们辛辛苦苦的,来回几个月,妈同哥哥商议商议,也该请一请,酬谢酬谢才是。不然,倒叫他们看着无礼似的。''

母女正说之间,见薛蟠自外而入,眼中尚有泪痕未干。一进门,便向他母亲拍手说道:``妈,可知道柳大哥、尤三姐的事么?''薛姨妈说:``我在园子里听见大家议论,正在这里才和你妹子说这件公案呢。''薛蟠道:``这事可奇不奇?''薛姨妈说:``可是柳相公那样一个年轻聪明的人,怎么就一时糊涂跟着道士去了呢?我想他前世必是有夙缘的有根基的人,所以才容易听得进这些度化他的话去。想你们相好了一场,他又无父母兄弟,只身一人在此,你也该各处找一找才是。靠那跛足道士疯疯癫癫的,能往那里远去!左不过在这房前左右的庙里寺里躲藏着罢咧。''薛蟠说:``何尝不是呢。我一听见这个信儿,就连忙带了小厮们在各处寻找去,连个影儿也没有。又去问人,人人都说不曾看见。我因如此,急的没法,唯有望着西北上大哭了一场回来了。''说着,眼圈儿又红上来了。薛姨妈说:``你既然找寻了没有,把你作朋友的心也尽了。焉知他这一出家,不是得了好处去呢?你也不必太过虑了。一则张罗张罗买卖,二则把你自己娶媳妇应办的事情,倒是早些料理料理。咱们家里没人手儿,竟是`笨雀儿先飞',省得临期丢三忘四的不齐全,令人笑话。再者,你妹妹才说,你也回家半个多月了,想货物也该发完了,同你作买卖去的伙计们,也该设桌酒席请请他们,酬酬劳乏才是。他们固然是咱家约请的吃工食劳金的人,到底也算是外客,又陪着你走了一二千里的路程,受了四五个月的辛苦,而且在路上又替你担了多少的惊怕沉重。''薛蟠闻听,说:``妈说的很是,妹妹想得周到。我也这样想来着,只因这些日子为各处发货,闹得头晕。又为柳大哥的亲事又忙了这几日,反倒落了一个空,白张罗了一会子,倒把正经事都误了。要不然,就定了明儿后儿下帖子请请罢。''薛姨妈道:``由你办去罢。''

话犹未了,外面小厮回说:``张管总的伙计着人送了两个箱子来,说这是爷各自买的,不在货账里面。本要早送来,因货物箱子压着,未得拿;昨日货物发完了,所以今儿才送来了。''一面说,一面又见两个小厮搬进了两个夹板夹的大棕箱来。薛蟠一见,说:``嗳哟,可是我怎么就糊涂到这一步田地了!特特的给妈和妹妹带来的东西都忘了,没拿了家里来,还是伙计送了来了。''宝钗说:``亏你才说还是特特的带来的,还是这样放了一二十日才送来,若不是`特特的'带来,必定是要放到年底下才送进来呢。你也诸事太不留心了。''薛蟠笑道:``想是我在路上叫贼人把魂吓掉了,还没归壳呢。''

说着,大家笑了一阵,便向回话的小厮说:``东西收下了,叫他们回去罢。''薛姨妈同宝钗忙问:``是什么好东西,这样捆着夹着的?''便命人挑了绳子,去了夹板,开了锁看时,却是些绸缎、绫锦、洋货等家常应用之物。独有宝钗他的那个箱子里,除了笔、墨、砚、各色笺纸、香袋、香珠、扇子、扇坠、花粉、胭脂、头油等物外,还有虎丘带来的自行人、酒令儿、水银灌的打筋斗的小小子,沙子灯,一出一出的泥人儿的戏,用青纱罩的匣子装着,又有在虎丘上作的薛蟠的像,泥捏成的与薛蟠毫无相差,以及许多碎小玩意儿的东西。宝钗一见,满心欢喜,便叫自己使的丫鬟来吩咐:``你将我的这个箱子与我拿了园子里去,我好就近从那边送送人。''说着,便起身来,告辞母亲,往园子里来了。这里薛姨妈将自己这个箱子里的东西取出,一分一分的打点清楚,着同喜丫头送往贾母并王夫人等处去不讲。

且说宝钗随着箱子到了自己房中,将东西逐件逐件的过了目,除将自己留用之外,遂一分一分配合妥当:也有送笔、墨、纸、砚的,也有送香袋、扇子、香坠的,也有送脂粉、头油的,也有单送玩意儿的;酌量其人分办。\elegantpar {只有黛玉的比别人不同,比众人加厚一倍。}{两人关系亲密}一一打点完毕,使莺儿同一个老婆子跟着,送往各处。

其李纨、宝玉等以及诸人,不过收了东西,赏赐来使,皆说些见面再谢等语而已。惟有林黛玉他见江南家乡之物,反自触物伤情,因想起他的父母来了。便对着这些东西,挥泪自叹,暗想:``我乃江南之人,父母双亡,又无兄弟,只身一人,可怜寄居外祖母家中,而且又多疾病,除外祖母以及舅母、姐妹看问外,那里还有一个姓林的亲人来看望看望,给我带些土物来。使我送送人,粧粧脸面也好。可见人若无至亲骨肉手足,是最寂寞、极冷清、极寒苦,没趣味的!''想到这里,不觉就大伤起心来了。紫鹃他乃伏侍黛玉多年,朝夕不离左右的,深知黛玉的心腹:他为见了江南故土之物,因感动了心怀,追思亲人的原故。但不敢说破,只在一旁劝说道:``姑娘的身子多病,早晚尚服丸药,这两日看着不过比那些日子略饮食好些,精神壮一点儿,还算不得十分大好。今儿宝姑娘送来这些东西,可见宝姑娘素日看姑娘甚重,姑娘看着该欢喜才是,为什么反倒伤感。这不是宝姑娘送东西为的是叫姑娘欢喜,这反倒是招姑娘烦恼了不成?若令宝姑娘知道了,怎么脸上下得来呢?再姑娘也要细想一想,老太太、太太们为姑娘的病症千方百计请好大夫诊脉配药调治,所为的是姑娘的病急好。这如今才好些,又这样哭哭啼啼的,岂不是自己糟蹋自己的身子,不肯叫老太太看着欢喜?难道说姑娘这个病,不是因素日从忧虑过度上伤多了气血得的么?姑娘的千金贵体别自己看轻了。''紫鹃正在这里劝解黛玉,只听见小丫头子在院内说:``宝二爷来了。''紫鹃忙说:``快请。''

话犹未毕,只见宝玉已进房来了。黛玉让坐毕,宝玉见黛玉泪痕满面,便问:``妹妹,又是谁得罪了你了?你两眼都哭得红了,是为什么?''黛玉不回答。旁边紫鹃将嘴向床里一扭,宝玉会意,便往床里一看,见堆着许多东西,就知是宝钗送来的,便笑着取笑说道:``好东西,想是妹妹要开杂货铺么?摆着这些东西作什么?''黛玉只是不理。紫鹃说:``二爷还提东西呢。因宝姑娘送了些东西来,我们姑娘一看,就伤心哭起来了。我正在这里好劝歹劝,总劝不住呢。而且又是才吃了饭,若只管哭,大发了,再吐了,犯了旧病,可不叫老太太骂死了我们么?倒是二爷来的很好,替我们劝一劝。''宝玉他本是聪明人,而且一心总留意在黛玉身上最重,所以深知黛玉之为人心细心窄,而又多心要强,不落人后,因见了人家哥哥自江南带了东西来送人,又系故乡之物,勾想起别的痛肠来,是以伤感是实。这是宝玉他心里揣摩黛玉的心病,却不肯明明说出,恐黛玉越发动情,乃笑道:``你们姑娘的原故不为别的,为的是宝姑娘送来的东西少,所以生气伤心。妹妹,你放心!等我明年往江南去,与你多多的带两船来,省得你淌眼抹泪的。''黛玉听了这话,不由``嗤''的一声笑了,忙说道:``我凭他怎么没见过世面,也到不了这一步田地上,因送的东西少,就生气伤心。我也不是两三岁的小孩子,你也忒把人看得平常小气了。我有我的原故,你那里知道。''说着说着,眼泪又流下来了。宝玉忙移至床上,挨黛玉坐下,将那些东西一件一件的拿起来,摆弄着细瞧,故意问:``这是什么,叫什么名字?那是怎么做的,这样齐整?这是什么,要他做什么使用?妹妹,你瞧,这一件可以摆在书阁儿上作陈设,那件放在条案上当古董儿倒好呢!''一味的将这些没要紧的话来支吾搭讪了一会,黛玉见宝玉那些呆样子,问东问西的,招人可笑,稍将烦恼丢开,略有些喜笑之意。宝玉见他有些喜色,便说道:``宝姐姐送东西来给咱们,我想着,咱们也该到他那里道个谢去才是,不知妹妹可去不去?''黛玉原不愿意为送些东西来就特特的道谢去,不过一时见了,说一声就完了。今被宝玉说得有理难以推托,无奈只得同宝玉去了。这且不提。

且说薛蟠听了母亲之言,急忙下请帖,置办酒筵。张罗了一日,果于次日,三四位伙计,俱各到齐。未免说了些店内发货、账目之事毕,列席让坐,薛蟠与各位奉酒酬劳。里面薛姨妈又着人出来致谢道乏,毕,内有一位问道:``今日席上怎么少柳大哥不出来?想是东家忘了,没请么?''薛蟠闻听,把眉一皱,叹了一口气,说道:``休提,休提,想来众位不知深情。若说起此人,真真可叹!于一二日前,忽被一个疯道士度化的出了家,跟着他去了。你们众位听一听,可奇不奇?''众人说道:``我们在店内也听见外面人吵嚷,说有一个道士三言两语把一个俗家子弟度了去了,又闻说一阵风刮了去了,又说驾着一片云彩去了,纷纷议论不一。我们也因发货事忙,那里有工夫当正经事,也没去细问细打听,到如今还是似信不信的。今听此言,那道士度化的原来就是柳大哥么?早知是他,我们大家也该劝解劝解。凭他怎么,也不容他去。嗳,又少了一个有趣儿的好朋友了!实实在在的可惜可叹。也怨不得东家你心里不爽快。想他那样一个伶俐人,未必是真跟了道士去罢。柳大哥他会些武艺,又有力量,或者看破了道士有些什么妖术邪法的破绽出来,故意假跟了他去,在背地里摆布他也未可知。''薛蟠说:``谁知道,果能如此,倒好罢咧,世上也少一个妖言惑众的人了。''众人道:``难道你知道了的时候,也没寻找他去不成?''薛蟠说:``城里城外,那里没有找到!因找了不见,不怕你们笑话,我还哭了一场呢。''言毕,只是长吁短叹,无精打彩的,不像往日高兴顽笑,让酒畅饮。席上虽设了些鸡鹅鱼鸭,山珍海味,美品佳肴,怎奈东家皱眉叹气,众伙计看此光景,不便久坐,不过随便喝了几钟酒,吃了些饭食,就都散了。这也不提。

且说宝玉拉了黛玉至宝钗处来道谢。彼此见面,未免说几句客言套语。黛玉便对宝钗说道:``大哥哥辛辛苦苦的能带了多少东西来,搁得住送我们这些处,你还剩什么呢?''宝玉说:``可是这话呢。''宝钗笑道:``东西不是什么好的,不过是远路带来的土物儿,大家看着略觉新鲜似的。我剩不剩什么要紧,我如今果爱什么,今年虽然不剩,明年我哥哥去时,再叫他给我带些个来,有什么难呢?''宝玉听说,忙笑道:``明年再带了什么来,我们还要姐姐送我们呢。可别忘了我们!''黛玉说:``你要,你只管说你要,不必拉扯上`我们'不`我们'的字眼,\elegantpar{姐姐瞧宝哥哥不是给姐姐来道谢,竟是又要定下明年的东西来了。}{总能劝人笑}姐姐瞧宝哥哥不是给姐姐来道谢,竟是又要定下明年的东西来了。''宝玉笑说:``我要出来,难道没有你一分儿不成?你不知道帮着说,反倒说起这散话来了。''大家听了,笑了一阵。宝钗问:``你二人如何来得这样巧,是谁会谁去的?''宝玉说:``休提,我因姐姐送我东西,想来林妹妹也必有,我想要来道谢,想林妹妹也必来道谢,故此我就到他房里会了他一同要到这里来。谁知到了他家,他正在屋里伤心落泪,也不知是为什么这样爱哭。''宝玉刚说到``落泪''两字,见黛玉瞪了他一眼,恐他往下还说。宝玉会意,随即便换过口来说道:``林妹妹这几日因身上不爽快,恐怕又病扳嘴,故此着急落泪。我劝解了一会子,才来了。一则道谢;二则省的一个人在房里坐着只管发闷。''宝钗说:``妹妹怕病闷,固然是正理,也不过是在那饮食起居、穿脱衣服冷热上加些小心就是了,为什么伤起心来呢?妹妹,你难道不知伤心难免不伤气血精神,把要紧的伤了,反倒要受病的罢咧。妹妹你细想想。''黛玉说:``姐姐说的很是。我何尝自己不知道呢,只因我这几年,姐姐是看见的,那一年不病一两场?病的我怕怕的了。见了药,吃了见效不见效,一闻见,先就头疼发恶心,怎么不叫我怕病呢?''宝钗说:``虽然如此说,却也不该伤心,倒是觉着身上不爽快,反自己勉强扎挣着出来,各处走走逛逛,把心松散松散,比在屋里闷坐着还强呢。伤心是自己添病的大毛病。我那两日不时觉着发懒,浑身乏倦,只是要歪着,心里也是为时气不好,怕病,因此偏扭着他,寻些事情作作,一般里也混过去了。妹妹别恼我说,越怕越有鬼。''宝玉听说,忙问道:``宝姐姐,鬼在那里呢?我怎么看不见一个儿?''惹得众人哄声大笑。宝钗道:``呆小爷,这是比喻的话,那里真有鬼呢!认真的果有鬼,你又该骇哭了。''黛玉因此笑道:``姐姐说的很是。很该说他,谁叫他嘴快!''宝玉说:``有人说我的不是,你就乐了。你这会子心里也不懊恼了,咱们也该走罢。''于是彼此又说笑了一回,二人辞了宝钗出来。宝玉仍把黛玉送至潇湘馆门首,自己回家。这且不提。

且说赵姨娘因见宝钗送环哥之物,忙忙接下,心中甚喜,满口夸奖:``人人都说宝姑娘会行事,很大方,今日看来,果然不错。他哥哥能带了多少东西来,他挨家送到,并不遗漏一处,也不露出谁薄谁厚,连我们搭拉嘴子,他都想到,实在的可敬。若是那林姑娘------也罢么,也没人给他送东西带什么来;即或有人带了来,\elegantpar{他也只是拣着那有势力、有体面的人头儿跟前才送去}{势利人有势利心,有势利眼},那里还临的到我们娘儿们身上呢!可见人会行事,真真的露着各别另样的好。''赵姨娘因环哥儿得了东西,深为得意,不住的托在掌上摆弄瞧看一会。想宝钗乃系王夫人之表侄女,特要在王夫人跟前卖好儿。自己叠叠歇歇的拿着那东西,走至王夫人房中,站在一旁说道:``这是他宝姑娘才给环哥他兄弟送来的。他年轻轻的人想的周到,我还给了送东西的小ㄚ头二百钱。听见说姨太太也给太太送来了,不知是什么东西?你们瞧瞧这一个门里头这就是两分儿,能有多少呢?怪不的老太太同太太都夸他疼他,果然招人爱。''说着,将抱的东西递过去与王夫人瞧,谁知王夫人头也没抬,手也没伸,只口内说了一声``好,给环哥儿玩罢咧'',并无正眼看一看。赵姨娘因招了一鼻子灰,满肚气恼,无精打彩的回至自己房中,将东西丢在一边,说了许多的劳儿三、巴儿四,不着要的一套闲话;也无人问他,他却自己咕嘟着嘴,一边子坐着。可见赵姨娘为人小器糊涂,饶得了东西,反说许多令人不入耳生厌的闲话,也怨不得\elegantpar{探春}{庶出日子难过}生气,看不起他。闲话休提。

且说宝钗送东西的ㄚ头回来,说:``也有道谢的,也有赏赐的,独有给巧姐儿的那一分儿,仍旧拿回来了。''宝钗一见,不知何意,便问:``为什么这一分儿没送去呢,还是送了去没收呢?''莺儿说:``我方才给环哥儿送东西的时候,见琏二奶奶往老太太房里去了。我想,琏二奶奶不在家,知道交给谁呢,所以没有送去。''宝钗说:``你也太糊涂了。二奶奶不在家,难道平儿、丰儿也不在家不成?你只管交给他们收下,等二奶奶回来,自有他们告诉就是了,必定要你当面交给才算么?''莺儿听了,复又拿着东西出了园子,往凤姐处去。在路上走着,便对拿东西的老婆子说:``早知道一就事儿送了去不完了,省得又跑这一趟。''老婆子说:``闲着也是白闲着,借此出来逛逛也好罢咧。只是姑娘你今日来回各处走了好些路儿,想是不惯,乏了,咱们送了这个,可就完了,一打总儿再歇着。''两人说着话,到了凤姐处,送了东西,回来见宝钗。

宝钗问道:``你见了琏二奶奶没有?''莺儿说:``我没有见。''宝钗说:``想是二奶奶还没回来么?''ㄚ头说:``回是回来了。因丰儿对我说:`二奶奶自老太太屋里回房来,不似往日欢天喜地的,一脸的怒气,叫了平儿去,唧唧咕咕的说话,也不叫人听见。连我都撵出来了,你不必去见,等我替你回一声儿就是了。'因此便着丰儿他拿进去,回了出来说:`二奶奶说,给你们姑娘道生受。'赏了我们一吊钱,我就回来了。''宝钗听了,自己纳了一会子闷,也想不出凤姐是为什么有气。这也不表。

且说袭人见宝玉回来,便问:``你怎么不逛就回来了?你原说约着林姑娘,你们两个同到宝姑娘处道谢去,可去了没有?''宝玉说:``你别问,我原说是要会林姑娘同去的,谁知到了他家,他在房里守着东西很很的不自在呢。我也知道林姑娘的那些原原故故的,又不好直问他,又不好说他,只装不知道儿,搭讪着说别的宽解了他一会子,才好了。然后方拉了他同到了宝姐姐那里道了谢,说了一会子闲话,方散了。我又送他到家,我才回来了。''袭人说:``你看送林姑娘的东西,比送你的是多是少,还是一样呢?''宝玉说:``比送我的多着一两倍呢。''袭人说:``这才是明白人,会行事。宝姑娘他想别的姊妹等都有亲的热的跟着,有人送东西,唯有林姑娘离家二三千里地远,又无有一个亲人在这里,那有人送东西。况且他们两个不但是亲戚,还是干姐妹,难道你不知道林姑娘去年曾认过薛姨太太作干妈的?论理多给他些也是该的。''

宝玉笑说:``你就是会评事的一个公道老儿。''说着话儿,便叫小丫头取了拐枕来,要在床上歪着。袭人说:``你不出去了?我有一句话告诉你。''宝玉便问:``什么话?''袭人说:``素日琏二奶奶待我很好,你是知道的。他自从病了一大场之后,如今又好了。我早就想着要到那里看看去,只因为琏二爷在家不方便,始终总没有去,闻说琏二爷不在家,你今日又不往那里去,而且初秋天气,不冷不热,一则看二奶奶,尽个礼,省得日后见了受他的数落;二则借此也逛一逛。你同他们看着家,我去去就来。''晴雯说:``这却是该的,难得这个巧空儿。''宝玉说:``我才为他议论宝姑娘,夸他是个公道人,这一件事行的,又是一个周到人了。''袭人笑道:``好小爷,你也不用夸我,你只在家同他们好生玩;好歹别睡觉,看睡出病来,又是我担沉重。''宝玉说:``我知道了,你只管去罢。''言毕,袭人遂到自己房里,换了两件新鲜衣服,拿着把儿镜照着,抿了抿头,匀了匀脸上脂粉,步出下房。复又嘱咐了晴雯、麝月几句话,便出了怡红院。

来至沁芳桥上立住,往四下里观看那园中景致。时值秋令,秋蝉鸣于树,草虫鸣于野;见这石榴花也开败了,荷叶也将残上来了,倒是芙蓉近着河边,都发了红铺铺的咕嘟子,衬着碧绿的叶儿,倒令人可爱。一壁里瞧着,一壁里下了桥。走了不远,迎见李纨房里使唤的丫头素云,跟着个老婆子,手里捧着一个洋漆盒儿走来。袭人便问:``往那里去?送的是什么东西?''素云说:``这是我们奶奶给三姑娘送去的菱角、鸡头。''袭人说:``这个东西,还是咱们园子里河内采的,还是外头买来的呢?''素云说:``这是我们房里使唤的刘妈妈,他告假瞧亲戚去带来的,孝敬奶奶。因三姑娘在我们那里坐着看见了,我们奶奶叫人剥了让他吃。他说:`才喝了热茶了,不吃,一会子再吃罢。'故此给三姑娘送了家去。''言毕,各自分路走了。

袭人远远的看见那边葡萄架底下,有一个人拿着掸子在那里动手动脚的,因迎着日光,看不真切。至离得不远,那祝老婆子见了袭人,便笑嘻嘻的迎上来,说道:``姑娘今日怎么得工夫出来闲逛,往那里去?''袭人说:``我那里还得工夫来逛,我往琏二奶奶家瞧瞧去。你在这里做什么呢?''那祝婆子说:``我在这里赶马蜂呢。今年三伏里的雨水少,不知怎么,这些果木树上长虫子,把果子吃得巴拉眼睛的,掉了好些下来,可惜了儿的白扔了!就是这葡萄,刚成了珠儿,怪好看的,那马蜂、蜜蜂儿满满的围着来蚛,都咬破了。这还罢了,喜鹊、雀儿,他也来吃这个葡萄。还有这样一个毛病儿,无论雀儿虫儿,一嘟噜上只咬破三五个,那破的水淌到好的上头,连这一嘟噜都是要烂的。这些雀儿、马蜂可恶着呢,故此我在这里赶。姑娘你瞧,咱们说话的空儿没赶,就蚛了许多上来了。''袭人道:``你就是不住手的赶,也赶不了许多;你刚赶了这里,那里又来了。倒是告诉买办说,叫他多多的作些冷布口袋来,一嘟噜一嘟噜的套上,免得翎禽草虫糟蹋,而且又透风,捂不坏。''婆子笑道:``倒是姑娘说的是。我今年才管上,那里就知道这些巧法儿呢。''

袭人说:``如今这园子里这些果品有好些种,到是那样先熟的快些?''老祝婆子说:``如今才入七月的门,果子都是才红上来,要是好吃,想来还得月尽头儿才熟透了呢。姑娘不信,我摘一个给姑娘尝尝。''袭人正色说道:``这那里使得?不但没熟吃不得,就是熟了,一则没有供鲜,二则主子们尚然没吃,咱们如何先吃得呢?你是这府里的陈人,难道连这个规矩也不晓得么?''老婆子忙笑道:``姑娘说得有理。我因为姑娘问我,我白这样说。''心内暗说道:``够了!我方才幸亏是在这里赶马蜂,若是顺着手儿摘一个尝尝,叫他看见,还了得了!''袭人说:``我方才告诉你要口袋的话,你就回一回二奶奶,叫管事的作去罢。''言毕,遂一直的出了园子的门,就到凤姐这里来了。

正是凤姐与平儿议论贾琏之事。因见袭人他是轻易不来之人,又不知是有什么事情,便连忙止住话语,勉强带笑说道:``贵人从那阵风儿刮了我们这个贱地来了?''袭人笑说:``我就知道奶奶见了我,是必定要先麻烦我一顿的,我有什么说的呢!但是奶奶欠安,本心惦着要过来请请安,头一件,琏二爷在家不便,二则奶奶在病中,又怕嫌烦,故未敢来。想奶奶素日疼爱我的那个分儿上,自必是体谅我,再不肯恼我的。''凤姐笑道:``宝兄弟屋里虽然人多,也就靠着你一个儿照看,也实在的离不开。我常听见平儿告诉我,说你背地里还惦着我,常问,我听见就狠喜欢的什么似的。今日见了你,我还要给你道谢呢,我还舍得麻烦你吗?我的姑娘!''袭人说:``我的奶奶,若是这样说,这就是真疼我了。''凤姐拉了袭人的手,让他坐下。袭人那里肯坐,让之再三,方在挨炕沿脚踏上坐了。

平儿忙自己端了茶来。袭人说:``你叫小人儿们端罢,劳动姑娘我倒不安。''一面站起,接过茶来吃着,一面回头看见床沿上放着一个活计簸罗儿,内装着一个大红洋锦的小兜肚,袭人说:``奶奶一天七事八事的,忙的不了,还有工夫作活计么?''凤姐说:``我本来就不会作什么,如今病了才好,又兼着家务事闹个不清,那里还有工夫做这些呢?要紧要紧的我都丢开了。这是我往老太太屋里请安去,正遇见薛姨太太送老太太这个锦,老太太说:`这个花红柳绿的,倒对给小孩子们做小衣小裳儿的,穿着倒好顽呢!'因此我就问老祖宗讨了来了。还惹的老祖宗说了好些顽话,说我是老太太的命中小人,见了什么要什么,见了什么拿什么。惹得众人都笑了。你是知道我是脸皮儿厚、不怕说的人,老祖宗只管说,我只管装听不见,拿着就走。所以才交给平儿,先给巧姐儿做件小兜肚穿着顽,剩下的等消闲有工夫再作别的。''

袭人听毕,笑道:``也就是奶奶,才能够怄的老祖宗喜欢罢咧。''伸手拿起来一看,便夸道:``果然好看!各样颜色都有。好材料也须得这样巧手的人做才对。况又是巧姐儿他穿的,抱了出去,谁不多看一看。''又问道:``巧姐儿那里去了?我怎么这半日没见他?''平儿说:``方才宝姑娘那里送了些顽的东西来,他一见了很希罕,就摆弄着顽了好一会子,他奶妈儿才抱了出去,想是乏了,睡觉去了。''袭人说:``巧姐儿比先前自然越发会顽了。''平儿说:``小脸蛋子吃得银盆似的,见了人就赶着笑,再不得罪人,真真是我奶奶的解闷的宝贝疙瘩儿。''凤姐便问:``宝兄弟在家作什么呢?''袭人笑道:``我才是求他同晴雯他们看家,我才告了假来了。可是呢!只顾说话,我也来了好大半天了,要回去了。别叫宝玉在家里抱怨,\elegantpar{说我屁股沉}{笑},到那里就坐住了。''说着,便立起身来告辞,回怡红院来了。这也不提。

且说凤姐见平儿送出袭人回来,复又把平儿叫入房中,追问前事,越说越气,说道:``二爷在外边偷娶老婆,你说你是听见二门上的小厮们说的。到底是那一个说的呢?''平儿说:``是旺儿他说的。''凤姐便命人把旺儿叫来,问道:``你二爷在外边买房子娶小老婆,你知道么?''旺儿说:``小的终日在二门上听差,如何知道二爷的事,这是听见兴儿告诉的。''凤姐说:``兴儿是几时告诉你的?''旺儿说:``还是二爷没起身的头里告诉我的。''凤姐又问:``兴儿在那里呢?''旺儿说:``兴儿在新二奶奶那里呢。''凤姐闻听,满腔怒气,啐了一口,骂道:``下作猴儿崽子!什么是`新奶奶'、`旧奶奶',你就私自封了奶奶了?满嘴里胡说,这就该打嘴巴。''又问:``兴儿他是跟二爷的人,他怎么没有跟了二爷去呢?''旺儿说:``特留下他在家里照看尤二姐,故此未曾跟了去。''凤姐听说,忙得一叠连声命旺儿:``快把兴儿叫了来!''

旺儿忙忙的跑了出去,见了兴儿只说:``二奶奶叫你呢。''兴儿正在外边同小人儿们顽笑,听见叫他,妙在也不问旺儿``二奶奶叫我做什么'',便跟了旺儿,急急忙忙的来至二门前。回明进去,见了凤姐,请了安,旁边侍立。凤姐一见,便先瞪了两眼,问道:``你们主子奴才在外面干的好事!你们打量我是呆瓜,不知道?你是紧跟二爷的人,自必深知根由。你须细细的对我实说,稍有一些儿隐瞒撒谎,我将你的腿打折了!''兴儿忙跪下磕头,说:``奶奶问的是什么事,是我同爷干的?''凤姐骂道:``好小杂种!你还敢来支吾我?我问你,二爷在外边,怎么就说成了尤二姐?怎么买房子、治家伙?怎么娶了过来?一五一十的说个明白,饶你的狗命!''

兴儿听说,仔细想了一想:``此事二府皆知,就是瞒着老爷、太太、老太太同二奶奶不知道,终久也是要知道的。我如今何苦来瞒着,不如告诉了他,省得挨眼前打,受委屈。''再兴儿一则年幼,不知事的轻重;二则素日又知道凤姐是个烈口子,连二爷还惧怕他五分;三则此事原是二爷同珍大爷、蓉哥他叔侄弟兄商量着办的,与自己无干。故此把主意想定,壮着胆子,跪下说道:``奶奶别生气,等奴才回禀奶奶听:只因那府里的大老爷的丧事上穿孝,不知二爷怎么看见过尤二姐几次,大约就看中了,动了要说的心。故此先同蓉哥商议,求蓉哥替二爷从中调停办理,作了媒人说合,事成之后,还许下谢候的礼。蓉哥满应,将此话转告诉了珍大爷;珍大爷告诉了珍大奶奶和尤老娘。尤老娘很愿意,但说是:`二姐从小儿已许过张家为媳,如何又许二爷呢?恐张家知道,生出事来不妥当。'珍大爷笑道:`这算什么大事,交给我!便说那张姓的小子,本是个穷苦破落户,那里见得多给他几两银子,叫他写张退亲的休书,就完了。'后来,果然找了姓张的来,如此说明,写了休书,给了银子去了。二爷闻知,才放心大胆的说定了。又恐怕奶奶知道,拦阻不依,所以在外边咱们后身儿买了几间房子,治了东西,就娶过来了。珍大爷还给了两口人使唤。二爷时常推说给老爷办事,又说给珍大爷张罗事,都是些支吾的谎话,竟是在外头住着。从前原是娘儿三个住着,还要商量给尤三姐说人家,又许下厚聘嫁他;如今尤三姐也死了,只剩下尤老娘跟着尤二姐住着作伴儿呢。这是一往从前的实话,并不敢隐瞒一句。''说毕,复又磕头。

凤姐听了这一篇言词,只气得痴呆了半天,面如金纸,\elegantpar{两只吊稍子眼越发直竖起来了,浑身乱战。}{气极气极}半晌,连话也说不上来,只是发怔。猛一低头,见兴儿在地下跪着,便说道:``这也没你的大不是,但只是二爷在外边行这样的事,你也该早些告诉我才是。这却很该打,因你肯实说,不撒谎,且饶恕你这一次。''兴儿说:``未能早回奶奶,这是奴才该死!''便叩头有声。凤姐说:``你去罢。''兴儿才立起身要走,凤姐又说:``叫你时,须要快来,不可远去。''兴儿连连答应了几个``是'',就出去了。到外面伸了伸舌头,说:``够了我的了,差一差儿没有捱一顿好打。''暗自后悔不该告诉旺儿,又愁二爷回来怎么见,各自害怕。这也不提。

且说凤姐见兴儿出去,回头向平儿说:``方才兴儿说的话,你都听见了没有?''平儿说:``我都听见了。''凤姐说:``天下那有这样没脸的男人!吃着碗里,看着锅里,见一个,爱一个,真成了喂不饱的狗,实在的是个弃旧迎新的坏货。只是可惜这五六品的顶戴给他!他别想着俗语说的`家花那有野花香'的话,他要信了这个话,可就大错了。多早晚在外面闹一个很没脸、亲戚朋友见不得的事出来,他才罢手呢!''平儿一旁劝道:``奶奶生气,却是该的。但奶奶身子才好了,也不可过于气恼。看二爷自从鲍二的女人那一件事之后,倒很收了心,好了呢,如今为什么又干起这样事来?这都是珍大爷他的不是。''凤姐说:``珍大爷固然有不是,也总因咱们那位下作不堪的\elegantpar{爷}{怪到我觉得青青喊丁鹏“爷”难受}他眼馋,人家才引诱他罢咧。俗语说的`牛不吃水,也强按头么?'''平儿说:``珍大爷干这样事,珍大奶奶也该拦着不依才是。''凤姐说:``可是这话咧!珍大奶奶也不想一想,把一个妹子要许几家子弟才好,先许了姓张的,今又嫁了姓贾的;天下的男人都死绝了,都嫁了贾家来!难道贾家的衣饭这样好不成?这不是说幸而那一个没脸的尤三姐知道好歹,早早儿的死了,若是不死,将来不是嫁宝玉,就是嫁环哥儿呢。总也不给那妹子留一些儿体面,叫妹子日后怎么抬头竖脸的见人呢?妹子好歹也罢咧!那妹子本来也不是他亲的,而且听见说原是个混帐烂桃。难道珍大奶奶现做着命妇,家中有这样一个打嘴现世的妹子,也不知道羞臊,躲避着些,反到大面儿上扬名打鼓的,在这门里丢丑,也不怕人笑话么?再者,珍大爷也是作官的人,别的律例不知道也罢了,连个服中娶亲、停妻再娶使不得的规矩,他也不知道不成?你替他细想一想,他干的这件事,是疼兄弟,还是害兄弟呢?''平儿说:``珍大爷只顾眼前,叫兄弟喜欢,也不管日后的轻重干系了。''凤姐儿冷笑道:``这是什么`叫兄弟喜欢',这是给他毒药吃呢!若论亲叔伯弟兄中,他年纪又最大,又居长,不知教导兄弟学好,反引诱兄弟学不长进,担罪名儿,日后闹出事来,他在一边缸沿儿上站着看热闹,真真我要骂也骂不出口来。\elegantpar{再者,他那边府里的丑事坏名儿,已经叫人听不上了,必定也叫兄弟学他一样,才好显不出他的丑来。}{贾珍看戏}这是什么作哥哥的道理?倒不如撒泡尿浸死了,替大老爷死了倒罢咧,活着作什么呢!你瞧东府里大老爷那样厚德,吃斋念佛行善,怎么反得了这样一个儿子孙子?大概是好风水都叫他老人家一个人全拔尽了。''平儿说:``想来不错。若不然,怎么这样差着格儿呢?''凤姐说:``这件事幸而老太太、老爷、太太不知道,倘或吹到这几位耳朵里去,不但咱们那没出息的二爷捱打受骂,就是珍大爷和珍大奶奶也保不住要吃不了要兜着走呢!''连说带詈,直闹了半天,连午饭也推头疼,没过去吃。

平儿看此光景越说越气,劝道:``奶奶也煞一煞气,事从缓来,等二爷回来,慢慢的再商量就是了。''凤姐听了此言,便从鼻孔内哼了两声,冷笑道:``好罢咧,等爷回来,可就迟了!''平儿便跪在地下,再三苦劝,安慰了一会子,凤姐才略消了些气恼。喝了口茶,喘息了良久,便要了拐枕,歪在床上,闭着眼睛打主意。平儿见凤姐儿躺着,方退出去。偏有不懂眼的几起子回事的人来,都被丰儿撵出去了。又有贾母处着玛瑙来问:``二奶奶为什么不吃饭?老太太不放心,着我来瞧来了。''凤姐知是贾母处打发人来,遂勉强起来,说:``我白有些头疼,并没别的病,请老太太放心。我已经躺了一躺儿,好了。''言毕,打发来人去后,却自己一个人将前事从头至尾细细的盘算多时,得了一个``一计害三贤''的狠主意出来。自己暗想:须得如此如此方妥。主意已定,也不告诉平儿,反外面作出嘻笑自若、无事的光景,并不露出恼恨妒嫉之意。

于是叫丫头传了来旺来吩咐,令他明日传唤匠役人等,收拾东厢房,裱糊铺设等语。平儿与众人皆不知为何缘故。要知端的,且看下回分解。

%{\href{../Text/part0071_split_000.html\#navto_1_a}{①}按:此回庚辰本缺。其他各本存在两种类型文字,且出入较大。列、戚、甲辰本此回情节安排完整合理,但较为罗嗦拖沓,玩其文字,当非出于曹雪芹手笔,或系脂砚等人据曹雪芹残稿补写而成。杨本及程甲乙本一系文字比较简练,显系经过后人整理,且存在删减过度及某些情节欠合理的问题。由于两种版本文字多寡悬殊,无法互校,故本书正文依据早出的列藏本,另将程甲本文字附录于下。}

