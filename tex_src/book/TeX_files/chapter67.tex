%{{第六十七回}}{第六十七回}}

\chapter{馈土物颦卿念故里\hspace{.5em}讯家童凤姐蓄阴谋}\footnote{按:此回庚辰本缺。其他各本存在两种类型文字,且出入较大。列、戚、甲辰本此回情节安排完整合理,但较为罗嗦拖沓,玩其文字,当非出于曹雪芹手笔,或系脂砚等人据曹雪芹残稿补写而成。杨本及程甲乙本一系文字比较简练,显系经过后人整理,且存在删减过度及某些情节欠合理的问题。由于两种版本文字多寡悬殊,无法互校,故本书正文依据早出的列藏本,另将程甲本文字附录于下。}
话说尤三姐自戕之后,尤老娘以及尤二姐、贾珍、尤氏并贾蓉、贾琏等闻之,俱各不胜悲恸伤感,自不必说,忙着人治买棺木盛殓,送往城外埋葬。却说柳湘莲见尤三姐身亡,迷性不悟,尚有痴情眷恋,被道人数句偈言打破迷关,竟自削发出家,跟随疯道人飘然而去,不知何往。后事暂且不表。

且说薛姨妈闻知湘莲已说定了尤三姐为妻,心甚喜悦,正自高高兴兴要打算替他买房屋、治器用、办妆奁,择吉日迎娶过门等事,以报他救命之恩。忽有家中小厮见薛姨妈,告知尤三姐自戕与柳湘莲出家的信息,心甚叹息。正自猜疑是为什么原故,时值宝钗从园子里过来,薛姨妈便对宝钗说道:“我的儿,你听见了没有?你珍大嫂子的妹妹尤三姐,他不是已经许定了给你哥哥的义弟柳湘莲的?这也很好。不知为什么尤三姐自刎了,柳湘莲也出了家了。真正奇怪的事,叫人意想不到!”宝钗听了,并不在意,便说道:“俗语说的好,‘天有不测风云,人有旦夕祸福’。这也是他们前生命定,活该不是夫妻。妈所为的是因有救哥哥的一段好处,故谆谆感叹。如果他两人齐齐全全的,妈自然该替他料理,如今死的死了,出家的出了家了,依我说,也只好由他罢了。妈也不必为他们伤感,损了自己的身子。倒是自从哥哥起江南回来了一二十日,贩了来的货物,想来也该发完了,那同伴去的伙计们辛辛苦苦的,来回几个月,妈同哥哥商议商议,也该请一请,酬谢酬谢才是。不然,倒叫他们看着无礼似的。”

母女正说之间,见薛蟠自外而入,眼中尚有泪痕未干。一进门,便向他母亲拍手说道:“妈,可知道柳大哥、尤三姐的事么?”薛姨妈说:“我在园子里听见大家议论,正在这里才和你妹子说这件公案呢。”薛蟠道:“这事可奇不奇?”薛姨妈说:“可是柳相公那样一个年轻聪明的人,怎么就一时糊涂跟着道士去了呢?我想他前世必是有夙缘的有根基的人,所以才容易听得进这些度化他的话去。想你们相好了一场,他又无父母兄弟,只身一人在此,你也该各处找一找才是。靠那跛足道士疯疯癫癫的,能往那里远去!左不过在这房前左右的庙里寺里躲藏着罢咧。”薛蟠说:“何尝不是呢。我一听见这个信儿,就连忙带了小厮们在各处寻找去,连个影儿也没有。又去问人,人人都说不曾看见。我因如此,急的没法,唯有望着西北上大哭了一场回来了。”说着,眼圈儿又红上来了。薛姨妈说:“你既然找寻了没有,把你作朋友的心也尽了。焉知他这一出家,不是得了好处去呢?你也不必太过虑了。一则张罗张罗买卖,二则把你自己娶媳妇应办的事情,倒是早些料理料理。咱们家里没人手儿,竟是‘笨雀儿先飞’,省得临期丢三忘四的不齐全,令人笑话。再者,你妹妹才说,你也回家半个多月了,想货物也该发完了,同你作买卖去的伙计们,也该设桌酒席请请他们,酬酬劳乏才是。他们固然是咱家约请的吃工食劳金的人,到底也算是外客,又陪着你走了一二千里的路程,受了四五个月的辛苦,而且在路上又替你担了多少的惊怕沉重。”薛蟠闻听,说:“妈说的很是,妹妹想得周到。我也这样想来着,只因这些日子为各处发货,闹得头晕。又为柳大哥的亲事又忙了这几日,反倒落了一个空,白张罗了一会子,倒把正经事都误了。要不然,就定了明儿后儿下帖子请请罢。”薛姨妈道:“由你办去罢。”

话犹未了,外面小厮回说:“张管总的伙计着人送了两个箱子来,说这是爷各自买的,不在货账里面。本要早送来,因货物箱子压着,未得拿;昨日货物发完了,所以今儿才送来了。”一面说,一面又见两个小厮搬进了两个夹板夹的大棕箱来。薛蟠一见,说:“嗳哟,可是我怎么就糊涂到这一步田地了!特特的给妈和妹妹带来的东西都忘了,没拿了家里来,还是伙计送了来了。”宝钗说:“亏你才说还是特特的带来的,还是这样放了一二十日才送来,若不是‘特特的’带来,必定是要放到年底下才送进来呢。你也诸事太不留心了。”薛蟠笑道:“想是我在路上叫贼人把魂吓掉了,还没归壳呢。”

说着,大家笑了一阵,便向回话的小厮说:“东西收下了,叫他们回去罢。”薛姨妈同宝钗忙问:“是什么好东西,这样捆着夹着的?”便命人挑了绳子,去了夹板,开了锁看时,却是些绸缎、绫锦、洋货等家常应用之物。独有宝钗他的那个箱子里,除了笔、墨、砚、各色笺纸、香袋、香珠、扇子、扇坠、花粉、胭脂、头油等物外,还有虎丘带来的自行人、酒令儿、水银灌的打筋斗的小小子,沙子灯,一出一出的泥人儿的戏,用青纱罩的匣子装着,又有在虎丘上作的薛蟠的像,泥捏成的与薛蟠毫无相差,以及许多碎小玩意儿的东西。宝钗一见,满心欢喜,便叫自己使的丫鬟来吩咐:“你将我的这个箱子与我拿了园子里去,我好就近从那边送送人。”说着,便起身来,告辞母亲,往园子里来了。这里薛姨妈将自己这个箱子里的东西取出,一分一分的打点清楚,着同喜丫头送往贾母并王夫人等处去不讲。

且说宝钗随着箱子到了自己房中,将东西逐件逐件的过了目,除将自己留用之外,遂一分一分配合妥当:也有送笔、墨、纸、砚的,也有送香袋、扇子、香坠的,也有送脂粉、头油的,也有单送玩意儿的;酌量其人分办。只有黛玉的比别人不同,比众人加厚一倍。一一打点完毕,使莺儿同一个老婆子跟着,送往各处。

其李纨、宝玉等以及诸人,不过收了东西,赏赐来使,皆说些见面再谢等语而已。惟有林黛玉他见江南家乡之物,反自触物伤情,因想起他的父母来了。便对着这些东西,挥泪自叹,暗想:“我乃江南之人,父母双亡,又无兄弟,只身一人,可怜寄居外祖母家中,而且又多疾病,除外祖母以及舅母、姐妹看问外,那里还有一个姓林的亲人来看望看望,给我带些土物来。使我送送人,粧粧脸面也好。可见人若无至亲骨肉手足,是最寂寞、极冷清、极寒苦,没趣味的!”想到这里,不觉就大伤起心来了。紫鹃他乃伏侍黛玉多年,朝夕不离左右的,深知黛玉的心腹:他为见了江南故土之物,因感动了心怀,追思亲人的原故。但不敢说破,只在一旁劝说道:“姑娘的身子多病,早晚尚服丸药,这两日看着不过比那些日子略饮食好些,精神壮一点儿,还算不得十分大好。今儿宝姑娘送来这些东西,可见宝姑娘素日看姑娘甚重,姑娘看着该欢喜才是,为什么反倒伤感。这不是宝姑娘送东西为的是叫姑娘欢喜,这反倒是招姑娘烦恼了不成?若令宝姑娘知道了,怎么脸上下得来呢?再姑娘也要细想一想,老太太、太太们为姑娘的病症千方百计请好大夫诊脉配药调治,所为的是姑娘的病急好。这如今才好些,又这样哭哭啼啼的,岂不是自己糟蹋自己的身子,不肯叫老太太看着欢喜?难道说姑娘这个病,不是因素日从忧虑过度上伤多了气血得的么?姑娘的千金贵体别自己看轻了。”紫鹃正在这里劝解黛玉,只听见小丫头子在院内说:“宝二爷来了。”紫鹃忙说:“快请。”

话犹未毕,只见宝玉已进房来了。黛玉让坐毕,宝玉见黛玉泪痕满面,便问:“妹妹,又是谁得罪了你了?你两眼都哭得红了,是为什么?”黛玉不回答。旁边紫鹃将嘴向床里一扭,宝玉会意,便往床里一看,见堆着许多东西,就知是宝钗送来的,便笑着取笑说道:“好东西,想是妹妹要开杂货铺么?摆着这些东西作什么?”黛玉只是不理。紫鹃说:“二爷还提东西呢。因宝姑娘送了些东西来,我们姑娘一看,就伤心哭起来了。我正在这里好劝歹劝,总劝不住呢。而且又是才吃了饭,若只管哭,大发了,再吐了,犯了旧病,可不叫老太太骂死了我们么?倒是二爷来的很好,替我们劝一劝。”宝玉他本是聪明人,而且一心总留意在黛玉身上最重,所以深知黛玉之为人心细心窄,而又多心要强,不落人后,因见了人家哥哥自江南带了东西来送人,又系故乡之物,勾想起别的痛肠来,是以伤感是实。这是宝玉他心里揣摩黛玉的心病,却不肯明明说出,恐黛玉越发动情,乃笑道:“你们姑娘的原故不为别的,为的是宝姑娘送来的东西少,所以生气伤心。妹妹,你放心!等我明年往江南去,与你多多的带两船来,省得你淌眼抹泪的。”黛玉听了这话,不由“嗤”的一声笑了,忙说道:“我凭他怎么没见过世面,也到不了这一步田地上,因送的东西少,就生气伤心。我也不是两三岁的小孩子,你也忒把人看得平常小气了。我有我的原故,你那里知道。”说着说着,眼泪又流下来了。宝玉忙移至床上,挨黛玉坐下,将那些东西一件一件的拿起来,摆弄着细瞧,故意问:“这是什么,叫什么名字?那是怎么做的,这样齐整?这是什么,要他做什么使用?妹妹,你瞧,这一件可以摆在书阁儿上作陈设,那件放在条案上当古董儿倒好呢!”一味的将这些没要紧的话来支吾搭讪了一会,黛玉见宝玉那些呆样子,问东问西的,招人可笑,稍将烦恼丢开,略有些喜笑之意。宝玉见他有些喜色,便说道:“宝姐姐送东西来给咱们,我想着,咱们也该到他那里道个谢去才是,不知妹妹可去不去?”黛玉原不愿意为送些东西来就特特的道谢去,不过一时见了,说一声就完了。今被宝玉说得有理难以推托,无奈只得同宝玉去了。这且不提。

且说薛蟠听了母亲之言,急忙下请帖,置办酒筵。张罗了一日,果于次日,三四位伙计,俱各到齐。未免说了些店内发货、账目之事毕,列席让坐,薛蟠与各位奉酒酬劳。里面薛姨妈又着人出来致谢道乏,毕,内有一位问道:“今日席上怎么少柳大哥不出来?想是东家忘了,没请么?”薛蟠闻听,把眉一皱,叹了一口气,说道:“休提,休提,想来众位不知深情。若说起此人,真真可叹!于一二日前,忽被一个疯道士度化的出了家,跟着他去了。你们众位听一听,可奇不奇?”众人说道:“我们在店内也听见外面人吵嚷,说有一个道士三言两语把一个俗家子弟度了去了,又闻说一阵风刮了去了,又说驾着一片云彩去了,纷纷议论不一。我们也因发货事忙,那里有工夫当正经事,也没去细问细打听,到如今还是似信不信的。今听此言,那道士度化的原来就是柳大哥么?早知是他,我们大家也该劝解劝解。凭他怎么,也不容他去。嗳,又少了一个有趣儿的好朋友了!实实在在的可惜可叹。也怨不得东家你心里不爽快。想他那样一个伶俐人,未必是真跟了道士去罢。柳大哥他会些武艺,又有力量,或者看破了道士有些什么妖术邪法的破绽出来,故意假跟了他去,在背地里摆布他也未可知。”薛蟠说:“谁知道,果能如此,倒好罢咧,世上也少一个妖言惑众的人了。”众人道:“难道你知道了的时候,也没寻找他去不成?”薛蟠说:“城里城外,那里没有找到!因找了不见,不怕你们笑话,我还哭了一场呢。”言毕,只是长吁短叹,无精打彩的,不像往日高兴顽笑,让酒畅饮。席上虽设了些鸡鹅鱼鸭,山珍海味,美品佳肴,怎奈东家皱眉叹气,众伙计看此光景,不便久坐,不过随便喝了几钟酒,吃了些饭食,就都散了。这也不提。

且说宝玉拉了黛玉至宝钗处来道谢。彼此见面,未免说几句客言套语。黛玉便对宝钗说道:“大哥哥辛辛苦苦的能带了多少东西来,搁得住送我们这些处,你还剩什么呢?”宝玉说:“可是这话呢。”宝钗笑道:“东西不是什么好的,不过是远路带来的土物儿,大家看着略觉新鲜似的。我剩不剩什么要紧,我如今果爱什么,今年虽然不剩,明年我哥哥去时,再叫他给我带些个来,有什么难呢?”宝玉听说,忙笑道:“明年再带了什么来,我们还要姐姐送我们呢。可别忘了我们!”黛玉说:“你要,你只管说你要,不必拉扯上‘我们’不‘我们’的字眼,|姐姐瞧宝哥哥不是给姐姐来道谢,竟是又要定下明年的东西来了。姐姐瞧宝哥哥不是给姐姐来道谢,竟是又要定下明年的东西来了。”宝玉笑说:“我要出来,难道没有你一分儿不成?你不知道帮着说,反倒说起这散话来了。”大家听了,笑了一阵。宝钗问:“你二人如何来得这样巧,是谁会谁去的?”宝玉说:“休提,我因姐姐送我东西,想来林妹妹也必有,我想要来道谢,想林妹妹也必来道谢,故此我就到他房里会了他一同要到这里来。谁知到了他家,他正在屋里伤心落泪,也不知是为什么这样爱哭。”宝玉刚说到“落泪”两字,见黛玉瞪了他一眼,恐他往下还说。宝玉会意,随即便换过口来说道:“林妹妹这几日因身上不爽快,恐怕又病扳嘴,故此着急落泪。我劝解了一会子,才来了。一则道谢;二则省的一个人在房里坐着只管发闷。”宝钗说:“妹妹怕病闷,固然是正理,也不过是在那饮食起居、穿脱衣服冷热上加些小心就是了,为什么伤起心来呢?妹妹,你难道不知伤心难免不伤气血精神,把要紧的伤了,反倒要受病的罢咧。妹妹你细想想。”黛玉说:“姐姐说的很是。我何尝自己不知道呢,只因我这几年,姐姐是看见的,那一年不病一两场?病的我怕怕的了。见了药,吃了见效不见效,一闻见,先就头疼发恶心,怎么不叫我怕病呢?”宝钗说:“虽然如此说,却也不该伤心,倒是觉着身上不爽快,反自己勉强扎挣着出来,各处走走逛逛,把心松散松散,比在屋里闷坐着还强呢。伤心是自己添病的大毛病。我那两日不时觉着发懒,浑身乏倦,只是要歪着,心里也是为时气不好,怕病,因此偏扭着他,寻些事情作作,一般里也混过去了。妹妹别恼我说,越怕越有鬼。”宝玉听说,忙问道:“宝姐姐,鬼在那里呢?我怎么看不见一个儿?”惹得众人哄声大笑。宝钗道:“呆小爷,这是比喻的话,那里真有鬼呢!认真的果有鬼,你又该骇哭了。”黛玉因此笑道:“姐姐说的很是。很该说他,谁叫他嘴快!”宝玉说:“有人说我的不是,你就乐了。你这会子心里也不懊恼了,咱们也该走罢。”于是彼此又说笑了一回,二人辞了宝钗出来。宝玉仍把黛玉送至潇湘馆门首,自己回家。这且不提。

且说赵姨娘因见宝钗送环哥之物,忙忙接下,心中甚喜,满口夸奖:“人人都说宝姑娘会行事,很大方,今日看来,果然不错。他哥哥能带了多少东西来,他挨家送到,并不遗漏一处,也不露出谁薄谁厚,连我们搭拉嘴子,他都想到,实在的可敬。若是那林姑娘------也罢么,也没人给他送东西带什么来;即或有人带了来,他也只是拣着那有势力、有体面的人头儿跟前才送去,那里还临的到我们娘儿们身上呢!可见人会行事,真真的露着各别另样的好。”赵姨娘因环哥儿得了东西,深为得意,不住的托在掌上摆弄瞧看一会。想宝钗乃系王夫人之表侄女,特要在王夫人跟前卖好儿。自己叠叠歇歇的拿着那东西,走至王夫人房中,站在一旁说道:“这是他宝姑娘才给环哥他兄弟送来的。他年轻轻的人想的周到,我还给了送东西的小ㄚ头二百钱。听见说姨太太也给太太送来了,不知是什么东西?你们瞧瞧这一个门里头这就是两分儿,能有多少呢?怪不的老太太同太太都夸他疼他,果然招人爱。”说着,将抱的东西递过去与王夫人瞧,谁知王夫人头也没抬,手也没伸,只口内说了一声“好,给环哥儿玩罢咧”,并无正眼看一看。赵姨娘因招了一鼻子灰,满肚气恼,无精打彩的回至自己房中,将东西丢在一边,说了许多的劳儿三、巴儿四,不着要的一套闲话;也无人问他,他却自己咕嘟着嘴,一边子坐着。可见赵姨娘为人小器糊涂,饶得了东西,反说许多令人不入耳生厌的闲话,也怨不得探春生气,看不起他。闲话休提。

且说宝钗送东西的ㄚ头回来,说:“也有道谢的,也有赏赐的,独有给巧姐儿的那一分儿,仍旧拿回来了。”宝钗一见,不知何意,便问:“为什么这一分儿没送去呢,还是送了去没收呢?”莺儿说:“我方才给环哥儿送东西的时候,见琏二奶奶往老太太房里去了。我想,琏二奶奶不在家,知道交给谁呢,所以没有送去。”宝钗说:“你也太糊涂了。二奶奶不在家,难道平儿、丰儿也不在家不成?你只管交给他们收下,等二奶奶回来,自有他们告诉就是了,必定要你当面交给才算么?”莺儿听了,复又拿着东西出了园子,往凤姐处去。在路上走着,便对拿东西的老婆子说:“早知道一就事儿送了去不完了,省得又跑这一趟。”老婆子说:“闲着也是白闲着,借此出来逛逛也好罢咧。只是姑娘你今日来回各处走了好些路儿,想是不惯,乏了,咱们送了这个,可就完了,一打总儿再歇着。”两人说着话,到了凤姐处,送了东西,回来见宝钗。

宝钗问道:“你见了琏二奶奶没有?”莺儿说:“我没有见。”宝钗说:“想是二奶奶还没回来么?”ㄚ头说:“回是回来了。因丰儿对我说:‘二奶奶自老太太屋里回房来,不似往日欢天喜地的,一脸的怒气,叫了平儿去,唧唧咕咕的说话,也不叫人听见。连我都撵出来了,你不必去见,等我替你回一声儿就是了。’因此便着丰儿他拿进去,回了出来说:‘二奶奶说,给你们姑娘道生受。’赏了我们一吊钱,我就回来了。”宝钗听了,自己纳了一会子闷,也想不出凤姐是为什么有气。这也不表。

且说袭人见宝玉回来,便问:“你怎么不逛就回来了?你原说约着林姑娘,你们两个同到宝姑娘处道谢去,可去了没有?”宝玉说:“你别问,我原说是要会林姑娘同去的,谁知到了他家,他在房里守着东西很很的不自在呢。我也知道林姑娘的那些原原故故的,又不好直问他,又不好说他,只装不知道儿,搭讪着说别的宽解了他一会子,才好了。然后方拉了他同到了宝姐姐那里道了谢,说了一会子闲话,方散了。我又送他到家,我才回来了。”袭人说:“你看送林姑娘的东西,比送你的是多是少,还是一样呢?”宝玉说:“比送我的多着一两倍呢。”袭人说:“这才是明白人,会行事。宝姑娘他想别的姊妹等都有亲的热的跟着,有人送东西,唯有林姑娘离家二三千里地远,又无有一个亲人在这里,那有人送东西。况且他们两个不但是亲戚,还是干姐妹,难道你不知道林姑娘去年曾认过薛姨太太作干妈的?论理多给他些也是该的。”

宝玉笑说:“你就是会评事的一个公道老儿。”说着话儿,便叫小丫头取了拐枕来,要在床上歪着。袭人说:“你不出去了?我有一句话告诉你。”宝玉便问:“什么话?”袭人说:“素日琏二奶奶待我很好,你是知道的。他自从病了一大场之后,如今又好了。我早就想着要到那里看看去,只因为琏二爷在家不方便,始终总没有去,闻说琏二爷不在家,你今日又不往那里去,而且初秋天气,不冷不热,一则看二奶奶,尽个礼,省得日后见了受他的数落;二则借此也逛一逛。你同他们看着家,我去去就来。”晴雯说:“这却是该的,难得这个巧空儿。”宝玉说:“我才为他议论宝姑娘,夸他是个公道人,这一件事行的,又是一个周到人了。”袭人笑道:“好小爷,你也不用夸我,你只在家同他们好生玩;好歹别睡觉,看睡出病来,又是我担沉重。”宝玉说:“我知道了,你只管去罢。”言毕,袭人遂到自己房里,换了两件新鲜衣服,拿着把儿镜照着,抿了抿头,匀了匀脸上脂粉,步出下房。复又嘱咐了晴雯、麝月几句话,便出了怡红院。

来至沁芳桥上立住,往四下里观看那园中景致。时值秋令,秋蝉鸣于树,草虫鸣于野;见这石榴花也开败了,荷叶也将残上来了,倒是芙蓉近着河边,都发了红铺铺的咕嘟子,衬着碧绿的叶儿,倒令人可爱。一壁里瞧着,一壁里下了桥。走了不远,迎见李纨房里使唤的丫头素云,跟着个老婆子,手里捧着一个洋漆盒儿走来。袭人便问:“往那里去?送的是什么东西?”素云说:“这是我们奶奶给三姑娘送去的菱角、鸡头。”袭人说:“这个东西,还是咱们园子里河内采的,还是外头买来的呢?”素云说:“这是我们房里使唤的刘妈妈,他告假瞧亲戚去带来的,孝敬奶奶。因三姑娘在我们那里坐着看见了,我们奶奶叫人剥了让他吃。他说:‘才喝了热茶了,不吃,一会子再吃罢。’故此给三姑娘送了家去。”言毕,各自分路走了。

袭人远远的看见那边葡萄架底下,有一个人拿着掸子在那里动手动脚的,因迎着日光,看不真切。至离得不远,那祝老婆子见了袭人,便笑嘻嘻的迎上来,说道:“姑娘今日怎么得工夫出来闲逛,往那里去?”袭人说:“我那里还得工夫来逛,我往琏二奶奶家瞧瞧去。你在这里做什么呢?”那祝婆子说:“我在这里赶马蜂呢。今年三伏里的雨水少,不知怎么,这些果木树上长虫子,把果子吃得巴拉眼睛的,掉了好些下来,可惜了儿的白扔了!就是这葡萄,刚成了珠儿,怪好看的,那马蜂、蜜蜂儿满满的围着来蚛,都咬破了。这还罢了,喜鹊、雀儿,他也来吃这个葡萄。还有这样一个毛病儿,无论雀儿虫儿,一嘟噜上只咬破三五个,那破的水淌到好的上头,连这一嘟噜都是要烂的。这些雀儿、马蜂可恶着呢,故此我在这里赶。姑娘你瞧,咱们说话的空儿没赶,就蚛了许多上来了。”袭人道:“你就是不住手的赶,也赶不了许多;你刚赶了这里,那里又来了。倒是告诉买办说,叫他多多的作些冷布口袋来,一嘟噜一嘟噜的套上,免得翎禽草虫糟蹋,而且又透风,捂不坏。”婆子笑道:“倒是姑娘说的是。我今年才管上,那里就知道这些巧法儿呢。”

袭人说:“如今这园子里这些果品有好些种,到是那样先熟的快些?”老祝婆子说:“如今才入七月的门,果子都是才红上来,要是好吃,想来还得月尽头儿才熟透了呢。姑娘不信,我摘一个给姑娘尝尝。”袭人正色说道:“这那里使得?不但没熟吃不得,就是熟了,一则没有供鲜,二则主子们尚然没吃,咱们如何先吃得呢?你是这府里的陈人,难道连这个规矩也不晓得么?”老婆子忙笑道:“姑娘说得有理。我因为姑娘问我,我白这样说。”心内暗说道:“够了!我方才幸亏是在这里赶马蜂,若是顺着手儿摘一个尝尝,叫他看见,还了得了!”袭人说:“我方才告诉你要口袋的话,你就回一回二奶奶,叫管事的作去罢。”言毕,遂一直的出了园子的门,就到凤姐这里来了。

正是凤姐与平儿议论贾琏之事。因见袭人他是轻易不来之人,又不知是有什么事情,便连忙止住话语,勉强带笑说道:“贵人从那阵风儿刮了我们这个贱地来了?”袭人笑说:“我就知道奶奶见了我,是必定要先麻烦我一顿的,我有什么说的呢!但是奶奶欠安,本心惦着要过来请请安,头一件,琏二爷在家不便,二则奶奶在病中,又怕嫌烦,故未敢来。想奶奶素日疼爱我的那个分儿上,自必是体谅我,再不肯恼我的。”凤姐笑道:“宝兄弟屋里虽然人多,也就靠着你一个儿照看,也实在的离不开。我常听见平儿告诉我,说你背地里还惦着我,常问,我听见就狠喜欢的什么似的。今日见了你,我还要给你道谢呢,我还舍得麻烦你吗?我的姑娘!”袭人说:“我的奶奶,若是这样说,这就是真疼我了。”凤姐拉了袭人的手,让他坐下。袭人那里肯坐,让之再三,方在挨炕沿脚踏上坐了。

平儿忙自己端了茶来。袭人说:“你叫小人儿们端罢,劳动姑娘我倒不安。”一面站起,接过茶来吃着,一面回头看见床沿上放着一个活计簸罗儿,内装着一个大红洋锦的小兜肚,袭人说:“奶奶一天七事八事的,忙的不了,还有工夫作活计么?”凤姐说:“我本来就不会作什么,如今病了才好,又兼着家务事闹个不清,那里还有工夫做这些呢?要紧要紧的我都丢开了。这是我往老太太屋里请安去,正遇见薛姨太太送老太太这个锦,老太太说:‘这个花红柳绿的,倒对给小孩子们做小衣小裳儿的,穿着倒好顽呢!’因此我就问老祖宗讨了来了。还惹的老祖宗说了好些顽话,说我是老太太的命中小人,见了什么要什么,见了什么拿什么。惹得众人都笑了。你是知道我是脸皮儿厚、不怕说的人,老祖宗只管说,我只管装听不见,拿着就走。所以才交给平儿,先给巧姐儿做件小兜肚穿着顽,剩下的等消闲有工夫再作别的。”

袭人听毕,笑道:“也就是奶奶,才能够怄的老祖宗喜欢罢咧。”伸手拿起来一看,便夸道:“果然好看!各样颜色都有。好材料也须得这样巧手的人做才对。况又是巧姐儿他穿的,抱了出去,谁不多看一看。”又问道:“巧姐儿那里去了?我怎么这半日没见他?”平儿说:“方才宝姑娘那里送了些顽的东西来,他一见了很希罕,就摆弄着顽了好一会子,他奶妈儿才抱了出去,想是乏了,睡觉去了。”袭人说:“巧姐儿比先前自然越发会顽了。”平儿说:“小脸蛋子吃得银盆似的,见了人就赶着笑,再不得罪人,真真是我奶奶的解闷的宝贝疙瘩儿。”凤姐便问:“宝兄弟在家作什么呢?”袭人笑道:“我才是求他同晴雯他们看家,我才告了假来了。可是呢!只顾说话,我也来了好大半天了,要回去了。别叫宝玉在家里抱怨,说我屁股沉,到那里就坐住了。”说着,便立起身来告辞,回怡红院来了。这也不提。

且说凤姐见平儿送出袭人回来,复又把平儿叫入房中,追问前事,越说越气,说道:“二爷在外边偷娶老婆,你说你是听见二门上的小厮们说的。到底是那一个说的呢?”平儿说:“是旺儿他说的。”凤姐便命人把旺儿叫来,问道:“你二爷在外边买房子娶小老婆,你知道么?”旺儿说:“小的终日在二门上听差,如何知道二爷的事,这是听见兴儿告诉的。”凤姐说:“兴儿是几时告诉你的?”旺儿说:“还是二爷没起身的头里告诉我的。”凤姐又问:“兴儿在那里呢?”旺儿说:“兴儿在新二奶奶那里呢。”凤姐闻听,满腔怒气,啐了一口,骂道:“下作猴儿崽子!什么是‘新奶奶’、‘旧奶奶’,你就私自封了奶奶了?满嘴里胡说,这就该打嘴巴。”又问:“兴儿他是跟二爷的人,他怎么没有跟了二爷去呢?”旺儿说:“特留下他在家里照看尤二姐,故此未曾跟了去。”凤姐听说,忙得一叠连声命旺儿:“快把兴儿叫了来!”

旺儿忙忙的跑了出去,见了兴儿只说:“二奶奶叫你呢。”兴儿正在外边同小人儿们顽笑,听见叫他,妙在也不问旺儿“二奶奶叫我做什么”,便跟了旺儿,急急忙忙的来至二门前。回明进去,见了凤姐,请了安,旁边侍立。凤姐一见,便先瞪了两眼,问道:“你们主子奴才在外面干的好事!你们打量我是呆瓜,不知道?你是紧跟二爷的人,自必深知根由。你须细细的对我实说,稍有一些儿隐瞒撒谎,我将你的腿打折了!”兴儿忙跪下磕头,说:“奶奶问的是什么事,是我同爷干的?”凤姐骂道:“好小杂种!你还敢来支吾我?我问你,二爷在外边,怎么就说成了尤二姐?怎么买房子、治家伙?怎么娶了过来?一五一十的说个明白,饶你的狗命!”

兴儿听说,仔细想了一想:“此事二府皆知,就是瞒着老爷、太太、老太太同二奶奶不知道,终久也是要知道的。我如今何苦来瞒着,不如告诉了他,省得挨眼前打,受委屈。”再兴儿一则年幼,不知事的轻重;二则素日又知道凤姐是个烈口子,连二爷还惧怕他五分;三则此事原是二爷同珍大爷、蓉哥他叔侄弟兄商量着办的,与自己无干。故此把主意想定,壮着胆子,跪下说道:“奶奶别生气,等奴才回禀奶奶听:只因那府里的大老爷的丧事上穿孝,不知二爷怎么看见过尤二姐几次,大约就看中了,动了要说的心。故此先同蓉哥商议,求蓉哥替二爷从中调停办理,作了媒人说合,事成之后,还许下谢候的礼。蓉哥满应,将此话转告诉了珍大爷;珍大爷告诉了珍大奶奶和尤老娘。尤老娘很愿意,但说是:‘二姐从小儿已许过张家为媳,如何又许二爷呢?恐张家知道,生出事来不妥当。’珍大爷笑道:‘这算什么大事,交给我!便说那张姓的小子,本是个穷苦破落户,那里见得多给他几两银子,叫他写张退亲的休书,就完了。’后来,果然找了姓张的来,如此说明,写了休书,给了银子去了。二爷闻知,才放心大胆的说定了。又恐怕奶奶知道,拦阻不依,所以在外边咱们后身儿买了几间房子,治了东西,就娶过来了。珍大爷还给了两口人使唤。二爷时常推说给老爷办事,又说给珍大爷张罗事,都是些支吾的谎话,竟是在外头住着。从前原是娘儿三个住着,还要商量给尤三姐说人家,又许下厚聘嫁他;如今尤三姐也死了,只剩下尤老娘跟着尤二姐住着作伴儿呢。这是一往从前的实话,并不敢隐瞒一句。”说毕,复又磕头。

凤姐听了这一篇言词,只气得痴呆了半天,面如金纸,两只吊稍子眼越发直竖起来了,浑身乱战。半晌,连话也说不上来,只是发怔。猛一低头,见兴儿在地下跪着,便说道:“这也没你的大不是,但只是二爷在外边行这样的事,你也该早些告诉我才是。这却很该打,因你肯实说,不撒谎,且饶恕你这一次。”兴儿说:“未能早回奶奶,这是奴才该死!”便叩头有声。凤姐说:“你去罢。”兴儿才立起身要走,凤姐又说:“叫你时,须要快来,不可远去。”兴儿连连答应了几个“是”,就出去了。到外面伸了伸舌头,说:“够了我的了,差一差儿没有捱一顿好打。”暗自后悔不该告诉旺儿,又愁二爷回来怎么见,各自害怕。这也不提。

且说凤姐见兴儿出去,回头向平儿说:“方才兴儿说的话,你都听见了没有?”平儿说:“我都听见了。”凤姐说:“天下那有这样没脸的男人!吃着碗里,看着锅里,见一个,爱一个,真成了喂不饱的狗,实在的是个弃旧迎新的坏货。只是可惜这五六品的顶戴给他!他别想着俗语说的‘家花那有野花香’的话,他要信了这个话,可就大错了。多早晚在外面闹一个很没脸、亲戚朋友见不得的事出来,他才罢手呢!”平儿一旁劝道:“奶奶生气,却是该的。但奶奶身子才好了,也不可过于气恼。看二爷自从鲍二的女人那一件事之后,倒很收了心,好了呢,如今为什么又干起这样事来?这都是珍大爷他的不是。”凤姐说:“珍大爷固然有不是,也总因咱们那位下作不堪的爷他眼馋,人家才引诱他罢咧。俗语说的‘牛不吃水,也强按头么?’”平儿说:“珍大爷干这样事,珍大奶奶也该拦着不依才是。”凤姐说:“可是这话咧!珍大奶奶也不想一想,把一个妹子要许几家子弟才好,先许了姓张的,今又嫁了姓贾的;天下的男人都死绝了,都嫁了贾家来!难道贾家的衣饭这样好不成?这不是说幸而那一个没脸的尤三姐知道好歹,早早儿的死了,若是不死,将来不是嫁宝玉,就是嫁环哥儿呢。总也不给那妹子留一些儿体面,叫妹子日后怎么抬头竖脸的见人呢?妹子好歹也罢咧!那妹子本来也不是他亲的,而且听见说原是个混帐烂桃。难道珍大奶奶现做着命妇,家中有这样一个打嘴现世的妹子,也不知道羞臊,躲避着些,反到大面儿上扬名打鼓的,在这门里丢丑,也不怕人笑话么?再者,珍大爷也是作官的人,别的律例不知道也罢了,连个服中娶亲、停妻再娶使不得的规矩,他也不知道不成?你替他细想一想,他干的这件事,是疼兄弟,还是害兄弟呢?”平儿说:“珍大爷只顾眼前,叫兄弟喜欢,也不管日后的轻重干系了。”凤姐儿冷笑道:“这是什么‘叫兄弟喜欢’,这是给他毒药吃呢!若论亲叔伯弟兄中,他年纪又最大,又居长,不知教导兄弟学好,反引诱兄弟学不长进,担罪名儿,日后闹出事来,他在一边缸沿儿上站着看热闹,真真我要骂也骂不出口来。再者,他那边府里的丑事坏名儿,已经叫人听不上了,必定也叫兄弟学他一样,才好显不出他的丑来。这是什么作哥哥的道理?倒不如撒泡尿浸死了,替大老爷死了倒罢咧,活着作什么呢!你瞧东府里大老爷那样厚德,吃斋念佛行善,怎么反得了这样一个儿子孙子?大概是好风水都叫他老人家一个人全拔尽了。”平儿说:“想来不错。若不然,怎么这样差着格儿呢?”凤姐说:“这件事幸而老太太、老爷、太太不知道,倘或吹到这几位耳朵里去,不但咱们那没出息的二爷捱打受骂,就是珍大爷和珍大奶奶也保不住要吃不了要兜着走呢!”连说带詈,直闹了半天,连午饭也推头疼,没过去吃。

平儿看此光景越说越气,劝道:“奶奶也煞一煞气,事从缓来,等二爷回来,慢慢的再商量就是了。”凤姐听了此言,便从鼻孔内哼了两声,冷笑道:“好罢咧,等爷回来,可就迟了!”平儿便跪在地下,再三苦劝,安慰了一会子,凤姐才略消了些气恼。喝了口茶,喘息了良久,便要了拐枕,歪在床上,闭着眼睛打主意。平儿见凤姐儿躺着,方退出去。偏有不懂眼的几起子回事的人来,都被丰儿撵出去了。又有贾母处着玛瑙来问:“二奶奶为什么不吃饭?老太太不放心,着我来瞧来了。”凤姐知是贾母处打发人来,遂勉强起来,说:“我白有些头疼,并没别的病,请老太太放心。我已经躺了一躺儿,好了。”言毕,打发来人去后,却自己一个人将前事从头至尾细细的盘算多时,得了一个“一计害三贤”的狠主意出来。自己暗想:须得如此如此方妥。主意已定,也不告诉平儿,反外面作出嘻笑自若、无事的光景,并不露出恼恨妒嫉之意。

于是叫丫头传了来旺来吩咐,令他明日传唤匠役人等,收拾东厢房,裱糊铺设等语。平儿与众人皆不知为何缘故。要知端的,且看下回分解。


第六十七回 见土仪颦卿思故里 闻秘事凤姐讯家童


话说尤三姐自尽之后,尤老娘和二姐儿、贾珍、贾琏等俱不胜悲恸,自不必说,忙令人盛殓,送往城外埋葬。柳湘莲见尤三姐身亡,痴情眷恋,却被道人数句冷言打破迷关,竟自截发出家,跟随疯道人飘然而去,不知何往。暂且不表。

  且说薛姨妈闻知湘莲已说定了尤三姐为妻,心中甚喜,正是高高兴兴要打算替他买房子,治家伙,择吉迎娶,以报他救命之恩。忽有家中小厮吵嚷“三姐儿自尽了”,被小丫头们听见,告知薛姨妈。薛姨妈不知为何,心甚叹息。正在猜疑,宝钗从园里过来,薛姨妈便对宝钗说道:“我的儿,你听见了没有?你珍大嫂子的妹妹三姑娘,他不是已经许定给你哥哥的义弟柳湘莲了么,不知为什么自刎了。那柳湘莲也不知往那里去了。真正奇怪的事,叫人意想不到。”宝钗听了,并不在意,便说道:“俗话说的好,‘天有不测风云,人有旦夕祸福’。这也是他们前生命定。前日妈妈为他救了哥哥,商量着替他料理,如今已经死的死了,走的走了,依我说,也只好由他罢了。妈妈也不必为他们伤感了。倒是自从哥哥打江南回来了一二十日,贩了来的货物,想来也该发完了,那同伴去的伙计们辛辛苦苦的,回来几个月了,妈妈和哥哥商议商议,也该请一请,酬谢酬谢才是。别叫人家看着无理似的。”

  母女正说话间,见薛蟠自外而入,眼中尚有泪痕。一进门来,便向他母亲拍手说道:“妈妈可知道柳二哥、尤三姐的事么?”薛姨妈说:“我才听见说,正在这里和你妹妹说这件公案呢。”薛蟠道:“妈妈可听见说柳湘莲跟着一个道士出了家了么?”薛姨妈道:“这越发奇了。怎么柳相公那样一个年轻的聪明人,一时糊涂,就跟着道士去了呢。我想你们好了一场,他又无父母兄弟,只身一人在此,你该各处找找他才是。靠那道士能往那里远去,左不过是在这方近左右的庙里寺里罢了。”薛蟠说:“何尝不是呢。我一听见这个信儿,就连忙带了小厮们在各处寻找,连一个影儿也没有。又去问人,都说没看见。”薛姨妈说:“你既找寻过没有,也算把你作朋友的心尽了。焉知他这一出家不是得了好处去呢。只是你如今也该张罗张罗买卖,二则把你自己娶媳妇应办的事情,倒早些料理料理。咱们家没人,俗语说的‘夯雀儿先飞’,省得临时丢三落四的不齐全,令人笑话。再者你妹妹才说,你也回家半个多月了,想货物也该发完了,同你去的伙计们,也该摆桌酒给他们道道乏才是。人家陪着你走了二三千里的路程,受了四五个月的辛苦,而且在路上又替你担了多少的惊怕沉重。”薛蟠听说,便道:“妈妈说的很是。倒是妹妹想的周到。我也这样想着,只因这些日子为各处发货闹的脑袋都大了。又为柳二哥的事忙了这几日,反倒落了一个空,白张罗了一会子,倒把正经事都误了。要不然定了明儿后儿下帖儿请罢。”薛姨妈道:“由你办去罢。”

  话犹未了,外面小厮进来回说:“管总的张大爷差人送了两箱子东西来,说这是爷各自买的,不在货账里面。本要早送来,因货物箱子压着,没得拿,昨儿货物发完了,所以今日才送来了。”一面说,一面又见两个小厮搬进了两个夹板夹的大棕箱。薛蟠一见,说:“嗳哟,可是我怎么就糊涂到这步田地了!特特的给妈和妹妹带来的东西,都忘了没拿了家里来,还是伙计送了来了。”宝钗说:“亏你说,还是特特的带来的才放了一二十天,若不是特特的带来,大约要放到年底下才送来呢。我看你也诸事太不留心了。”薛蟠笑道:“想是在路上叫人把魂吓掉了,还没归窍呢。”说着大家笑了一回,便向小丫头说:“出去告诉小厮们,东西收下,叫他们回去罢。”薛姨妈同宝钗因问:“到底是什么东西,这样捆着绑着的?”薛蟠便命叫两个小厮进来,解了绳子,去了夹板,开了锁看时,这一箱都是绸缎、绫锦、洋货等家常应用之物。薛蟠笑着道:“那一箱是给妹妹带的。”亲自来开。母女二人看时,却是些笔、墨、纸、砚、各色笺纸、香袋、香珠、扇子、扇坠、花粉、胭脂等物,外有虎丘带来的自行人、酒令儿、水银灌的打筋斗小小子、沙子灯、一出一出的泥人儿的戏,用青纱罩的匣子装着,又有在虎丘山上泥捏的薛蟠的小像,与薛蟠毫无相差。宝钗见了,别的都不理论,倒是薛蟠的小像,拿着细细看了一看,又看看他哥哥,不禁笑起来了。因叫莺儿带着几个老婆子将这些东西连箱子送到园里去,又和母亲哥哥说了一回闲话儿,才回园里去了。这里薛姨妈将箱子里的东西取出,一分一分的打点清楚,叫同喜送给贾母并王夫人等处不提。

  且说宝钗到了自己房中,将那些玩意儿一件一件的过了目,除了自己留用之外,一分一分配合妥当,也有送笔、墨、纸、砚的,也有送香袋、扇子、香坠的,也有送脂粉、头油的,有单送顽意儿的。只有黛玉的比别人不同,且又加厚一倍。一一打点完毕,使莺儿同着一个老婆子,跟着送往各处。

  这边姊妹诸人都收了东西,赏赐来使,说见面再谢。惟有林黛玉看见他家乡之物,反自触物伤情,想起父母双亡,又无兄弟,寄居亲戚家中,那里有人也给我带些土物?想到这里,不觉的又伤起心来了。紫鹃深知黛玉心肠,但也不敢说破,只在一旁劝道:“姑娘的身子多病,早晚服药,这两日看着比那些日子略好些。虽说精神长了一点儿,还算不得十分大好。今儿宝姑娘送来的这些东西,可见宝姑娘素日看得姑娘很重,姑娘看着该喜欢才是,为什么反倒伤起心来。这不是宝姑娘送东西来倒叫姑娘烦恼了不成?就是宝姑娘听见,反觉脸上不好看。再者这里老太太们为姑娘的病体,千方百计请好大夫配药诊治,也为是姑娘的病好。这如今才好些,又这样哭哭啼啼,岂不是自己遭塌了自己身子,叫老太太看着添了愁烦了么?况且姑娘这病,原是素日忧虑过度,伤了血气。姑娘的千金贵体,也别自己看轻了。”紫鹃正在这里劝解,只听见小丫头子在院内说:“宝二爷来了。”紫鹃忙说:“请二爷进来罢。”

  只见宝玉进房来了,黛玉让坐毕,宝玉见黛玉泪痕满面,便问:“妹妹,又是谁气着你了?”黛玉勉强笑道:“谁生什么气。”旁边紫鹃将嘴向床后桌上一努,宝玉会意,往那里一瞧,见堆着许多东西,就知道是宝钗送来的,便取笑说道:“那里这些东西,不是妹妹要开杂货铺啊?”黛玉也不答言。紫鹃笑着道:“二爷还提东西呢。因宝姑娘送了些东西来,姑娘一看就伤起心来了。我正在这里劝解,恰好二爷来的很巧,替我们劝劝。”宝玉明知黛玉是这个缘故,却也不敢提头儿,只得笑说道:“你们姑娘的缘故想来不为别的,必是宝姑娘送来的东西少,所以生气伤心。妹妹,你放心,等我明年叫人往江南去,与你多多的带两船来,省得你淌眼抹泪的。”

  黛玉听了这些话,也知宝玉是为自己开心,也不好推,也不好任,因说道:“我任凭怎么没见世面,也到不了这步田地,因送的东西少,就生气伤心。我又不是两三岁的小孩子,你也忒把人看得小气了。我有我的缘故,你那里知道。”说着,眼泪又流下来了。宝玉忙走到床前,挨着黛玉坐下,将那些东西一件一件拿起来摆弄着细瞧,故意问这是什么,叫什么名字;那是什么做的,这样齐整;这是什么,要他做什么使用。又说这一件可以摆在面前,又说那一件可以放在条桌上当古董儿倒好呢。一味的将些没要紧的话来厮混。黛玉见宝玉如此,自己心里倒过不去,便说:“你不用在这里混搅了。咱们到宝姐姐那边去罢。”宝玉巴不得黛玉出去散散闷,解了悲痛,便道:“宝姐姐送咱们东西,咱们原该谢谢去。”黛玉道:“自家姊妹,这倒不必。只是到他那边,薛大哥回来了,必然告诉他些南边的古迹儿,我去听听,只当回了家乡一趟的。”说着,眼圈儿又红了。宝玉便站着等他。黛玉只得同他出来,往宝钗那里去了。

  且说薛蟠听了母亲之言,急下了请帖,办了酒席。次日,请了四位伙计,俱已到齐,不免说些贩卖账目发货之事。不一时,上席让坐,薛蟠挨次斟了酒。薛姨妈又使人出来致意。大家喝着酒说闲话儿。内中一个道:“今日这席上短两个好朋友。”众人齐问是谁,那人道:“还有谁,就是贾府上的琏二爷和大爷的盟弟柳二爷。”大家果然都想起来,问着薛蟠道:“怎么不请琏二爷和柳二爷来?”薛蟠闻言,把眉一皱,叹口气道:“琏二爷又往平安州去了,头两天就起了身的。那柳二爷竟别提起,真是天下头一件奇事。什么是柳二爷,如今不知那里作柳道爷去了。”众人都诧异道:“这是怎么说?”薛蟠便把湘莲前后事体说了一遍。众人听了,越发骇异,因说道:“怪不的前日我们在店里仿仿佛佛也听见人吵嚷说,有一个道士三言两语把一个人度了去了,又说一阵风刮了去了。只不知是谁。我们正发货,那里有闲工夫打听这个事去,到如今还是似信不信的,谁知就是柳二爷呢。早知是他,我们大家也该劝他劝才是。任他怎么着,也不叫他去。”内中一个道:“别是这么着罢?”众人问怎么样,那人道:“柳二爷那样个伶俐人,未必是真跟了道士去罢。他原会些武艺,又有力量,或看破那道士的妖术邪法,特意跟他去,在背地摆布他,也未可知。”薛蟠道:“果然如此倒也罢了。世上这些妖言惑众的人,怎么没人治他一下子。”众人道:“那时难道你知道了也没找寻他去?”薛蟠说:“城里城外,那里没有找到?不怕你们笑话,我找不着他,还哭了一场呢。”言毕,只是长吁短叹无精打彩的,不像往日高兴。众伙计见他这样光景,自然不便久坐,不过随便喝了几杯酒,吃了饭,大家散了。

  且说宝玉同着黛玉到宝钗处来。宝玉见了宝钗,便说道:“大哥哥辛辛苦苦的带了东西来,姐姐留着使罢,又送我们。”宝钗笑道:“原不是什么好东西,不过是远路带来的土物儿,大家看着新鲜些就是了。”黛玉道:“这些东西我们小时候倒不理会,如今看见,真是新鲜物儿了。”宝钗因笑道:“妹妹知道,这就是俗语说的‘物离乡贵’,其实可算什么呢。”宝玉听了这话正对了黛玉方才的心事,连忙拿话岔道:“明年好歹大哥哥再去时,替我们多带些来。”黛玉瞅了他一眼,便道:“你要你只管说,不必拉扯上人。姐姐你瞧,宝哥哥不是给姐姐来道谢,竟又要定下明年的东西来了。”说的宝钗宝玉都笑了。三个人又闲话了一回,因提起黛玉的病来。宝钗劝了一回,因说道:“妹妹若觉着身子不爽快,倒要自己勉强拃挣着出来各处走走逛逛,散散心,比在屋里闷坐着到底好些。我那两日不是觉着发懒,浑身发热,只是要歪着,也因为时气不好,怕病,因此寻些事情自己混着。这两日才觉着好些了。”黛玉道:“姐姐说的何尝不是。我也是这么想着呢。”大家又坐了一会子方散。宝玉仍把黛玉送至潇湘馆门首,才各自回去了。

  且说赵姨娘因见宝钗送了贾环些东西,心中甚是喜欢,想道:“怨不得别人都说那宝丫头好,会做人,很大方,如今看起来果然不错。他哥哥能带了多少东西来,他挨门儿送到,并不遗漏一处,也不露出谁薄谁厚,连我们这样没时运的,他都想到了。若是那林丫头,他把我们娘儿们正眼也不瞧,那里还肯送我们东西?”一面想,一面把那些东西翻来覆去的摆弄瞧看一回。忽然想到宝钗系王夫人的亲戚,为何不到王夫人跟前卖个好儿呢。自己便蝎蝎螫螫的拿着东西,走至王夫人房中,站在旁边,陪笑说道:“这是宝姑娘才刚给环哥儿的。难为宝姑娘这么年轻的人,想的这么周到,真是大户人家的姑娘,又展样,又大方,怎么叫人不敬服呢。怪不得老太太和太太成日家都夸他疼他。我也不敢自专就收起来,特拿来给太太瞧瞧,太太也喜欢喜欢。”王夫人听了,早知道来意了,又见他说的不伦不类,也不便不理他,说道:“你自管收了去给环哥顽罢。”赵姨娘来时兴兴头头,谁知抹了一鼻子灰,满心生气,又不敢露出来,只得讪讪的出来了。到了自己房中,将东西丢在一边,嘴里咕咕哝哝自言自语道:“这个又算了个什么儿呢。”一面坐着,各自生了一回闷气。

  却说莺儿带着老婆子们送东西回来,回复了宝钗,将众人道谢的话并赏赐的银钱都回完了,那老婆子便出去了。莺儿走近前来一步,挨着宝钗悄悄的说道:“刚才我到琏二奶奶那边,看见二奶奶一脸的怒气。我送下东西出来时,悄悄的问小红,说刚才二奶奶从老太太屋里回来,不似往日欢天喜地的,叫了平儿去,唧唧咕咕的不知说了些什么。看那个光景,倒像有什么大事的似的。姑娘没听见那边老太太有什么事?”宝钗听了,也自己纳闷,想不出凤姐是为什么有气,便道:“各人家有各人的事,咱们那里管得。你去倒茶去罢。”莺儿于是出来,自去倒茶不提。

  且说宝玉送了黛玉回来,想着黛玉的孤苦,不免也替他伤感起来。因要将这话告诉袭人,进来时却只有麝月秋纹在房中。因问:“你袭人姐姐那里去了?”麝月道:“左不过在这几个院里,那里就丢了他。一时不见,就这样找。”宝玉笑着道:“不是怕丢了他。因我方才到林姑娘那边,见林姑娘又正伤心呢。问起来却是为宝姐姐送了他东西,他看见是他家乡的土物,不免对景伤情。我要告诉你袭人姐姐,叫他闲时过去劝劝。”正说着,晴雯进来了,因问宝玉道:“你回来了,你又要叫劝谁?”宝玉将方才的话说了一遍。晴雯道:“袭人姐姐才出去,听见他说要到琏二奶奶那边去。保不住还到林姑娘那里。”宝玉听了,便不言语。秋纹倒了茶来,宝玉漱了一口,递给小丫头子,心中着实不自在,就随便歪在床上。

  却说袭人因宝玉出门,自己作了回活计,忽想起凤姐身上不好,这几日也没有过去看看,况闻贾琏出门,正好大家说说话儿。便告诉晴雯:“好生在屋里,别都出去了,叫宝玉回来抓不着人。”晴雯道:“嗳哟,这屋里单你一个人记挂着他,我们都是白闲着混饭吃的。”袭人笑着,也不答言,就走了。

  刚来到沁芳桥畔,那时正是夏末秋初,池中莲藕新残相间,红绿离披。袭人走着,沿堤看顽了一回。猛抬头看见那边葡萄架底下有人拿着掸子在那里掸什么呢,走到跟前,却是老祝妈。那老婆子见了袭人,便笑嘻嘻的迎上来,说道:“姑娘怎么今日得工夫出来逛逛?”袭人道:“可不是。我要到琏二奶奶家瞧瞧去。你在这里做什么呢?”那婆子道:“我在这里赶蜜蜂儿。今年三伏里雨水少,这果子树上都有虫子,把果子吃的疤瘌流星的掉了好些下来。姑娘还不知道呢,这马蜂最可恶的,一嘟噜上只咬破三两个儿,那破的水滴到好的上头,连这一嘟噜都是要烂的。姑娘你瞧,咱们说话的空儿没赶,就落上许多了。”袭人道:“你就是不住手的赶,也赶不了许多。你倒是告诉买办,叫他多多做些小冷布口袋儿,一嘟噜套上一个,又透风,又不遭塌。”婆子笑道:“倒是姑娘说的是。我今年才管上,那里知道这个巧法儿呢。”因又笑着说道:“今年果子虽遭塌了些,味儿倒好,不信摘一个姑娘尝尝。”袭人正色道:“这那里使得。不但没熟吃不得,就是熟了,上头还没有供鲜,咱们倒先吃了。你是府里使老了的,难道连这个规矩都不懂了。”老祝忙笑道:“姑娘说得是。我见姑娘很喜欢,我才敢这么说,可就把规矩错了,我可是老糊涂了。”袭人道:“这也没有什么。只是你们有年纪的老奶奶们,别先领着头儿这么着就好了。”说着遂一径出了园门,来到凤姐这边。

  一到院里,只听凤姐说道:“天理良心,我在这屋里熬的越发成了贼了。”袭人听见这话,知道有原故了,又不好回来,又不好进去,遂把脚步放重些,隔着窗子问道:“平姐姐在家里呢么?”平儿忙答应着迎出来。袭人便问:“二奶奶也在家里呢么,身上可大安了?”说着,已走进来。凤姐装着在床上歪着呢,见袭人进来,也笑着站起来,说:“好些了,叫你惦着。怎么这几日不过我们这边坐坐?”袭人道:“奶奶身上欠安,本该天天过来请安才是。但只怕奶奶身上不爽快,倒要静静儿的歇歇儿,我们来了,倒吵的奶奶烦。”凤姐笑道:“烦是没的话。倒是宝兄弟屋里虽然人多,也就靠着你一个照看他,也实在的离不开。我常听见平儿告诉我,说你背地里还惦着我,常常问我。这就是你尽心了。”一面说着,叫平儿挪了张杌子放在床旁边,让袭人坐下。丰儿端进茶来,袭人欠身道:“妹妹坐着罢。”一面说闲话儿。只见一个小丫头子在外间屋里悄悄的和平儿说:“旺儿来了。在二门上伺候着呢。”又听见平儿也悄悄的道:“知道了。叫他先去,回来再来,别在门口儿站着。”袭人知他们有事,又说了两句话,便起身要走。凤姐道:“闲来坐坐,说说话儿,我倒开心。”因命平儿:“送送你妹妹。”平儿答应着送出来。只见两三个小丫头子,都在那里屏声息气齐齐的伺候着。袭人不知何事,便自去了。

  却说平儿送出袭人,进来回道:“旺儿才来了,因袭人在这里我叫他先到外头等等儿,这会子还是立刻叫他呢,还是等着?请奶奶的示下。”凤姐道:“叫他来。”平儿忙叫小丫头去传旺儿进来。这里凤姐又问平儿:“你到底是怎么听见说的?”平儿道:“就是头里那小丫头子的话。他说他在二门里头听见外头两个小厮说:‘这个新二奶奶比咱们旧二奶奶还俊呢,脾气儿也好。’不知是旺儿是谁,吆喝了两个一顿,说:‘什么新奶奶旧奶奶的,还不快悄悄儿的呢,叫里头知道了,把你的舌头还割了呢。’”平儿正说着,只见一个小丫头进来回说:“旺儿在外头伺候着呢。”凤姐听了,冷笑了一声说:“叫他进来。”那小丫头出来说:“奶奶叫呢。”旺儿连忙答应着进来。

  旺儿请了安,在外间门口垂手侍立。凤姐儿道:“你过来,我问你话。”旺儿才走到里间门旁站着。凤姐儿道:“你二爷在外头弄了人,你知道不知道?”旺儿又打着千儿回道:“奴才天天在二门上听差事,如何能知道二爷外头的事呢。”凤姐冷笑道:“你自然不知道。你要知道,你怎么拦人呢。”旺儿见这话,知道刚才的话已经走了风了,料着瞒不过,便又跪回道:“奴才实在不知。就是头里兴儿和喜儿两个人在那里混说,奴才吆喝了他们两句。内中深情底里奴才不知道,不敢妄回。求奶奶问兴儿,他是长跟二爷出门的。”凤姐听了,下死劲啐了一口,骂道:“你们这一起没良心的混账忘八崽子!都是一条藤儿,打量我不知道呢。先去给我把兴儿那个忘八崽子叫了来,你也不许走。问明白了他,回来再问你。好,好,好!这才是我使出来的好人呢!”那旺儿只得连声答应几个是,磕了个头爬起来出去,去叫兴儿。

  却说兴儿正在账房儿里和小厮们玩呢,听见说二奶奶叫,先唬了一跳,却也想不到是这件事发作了,连忙跟着旺儿进来。旺儿先进去,回说:“兴儿来了。”凤姐儿厉声道:“叫他!”那兴儿听见这个声音儿,早已没了主意了,只得乍着胆子进来。凤姐儿一见,便说:“好小子啊!你和你爷办的好事啊!你只实说罢!”兴儿一闻此言,又看见凤姐儿气色及两边丫头们的光景,早唬软了,不觉跪下,只是磕头。凤姐儿道:“论起这事来,我也听见说不与你相干。但只你不早来回我知道,这就是你的不是了。你要实说了,我还饶你,再有一字虚言,你先摸摸你腔子上几个脑袋瓜子!”兴儿战兢兢的朝上磕头道:“奶奶问的是什么事,奴才同爷办坏了?”凤姐听了,一腔火都发作起来,喝命:“打嘴巴!”旺儿过来才要打时,凤姐儿骂道:“什么糊涂忘八崽子!叫他自己打,用你打吗!一会子你再各人打你那嘴巴子还不迟呢。”那兴儿真个自己左右开弓打了自己十几个嘴巴。

  凤姐儿喝声“站住”,问道:“你二爷外头娶了什么新奶奶旧奶奶的事,你大概不知道啊。”兴儿见说出这件事来,越发着了慌,连忙把帽子抓下来在砖地上咕咚咕咚碰的头山响,口里说道:“只求奶奶超生,奴才再不敢撒一个字儿的谎。”凤姐道:“快说!”兴儿直蹶蹶的跪起来回道,“这事头里奴才也不知道。就是这一天,东府里大老爷送了殡,俞禄往珍大爷庙里去领银子。二爷同着蓉哥儿到了东府里,道儿上爷儿两个说起珍大奶奶那边的二位姨奶奶来。二爷夸他好,蓉哥儿哄着二爷,说把二姨奶奶说给二爷。”凤姐听到这里,使劲啐道:“呸,没脸的忘八蛋!他是你那一门子的姨奶奶!”兴儿忙又磕头说:“奴才该死!”往上瞅着,不敢言语。凤姐儿道:“完了吗?怎么不说了?”兴儿方才又回道:“奶奶恕奴才,奴才才敢回。”凤姐啐道:“放你妈的屁,这还什么恕不恕了。你好生给我往下说,好多着呢。”兴儿又回道:“二爷听见这个话就喜欢了。后来奴才也不知道怎么就弄真了。”

  凤姐微微冷笑道:“这个自然么,你可那里知道呢!你知道的只怕都烦了呢。是了,说底下的罢!”兴儿回道:“后来就是蓉哥儿给二爷找了房子。”凤姐忙问道:“如今房子在那里?”兴儿道:“就在府后头。”凤姐儿道:“哦。”回头瞅着平儿道:“咱们都是死人哪。你听听!”平儿也不敢作声。兴儿又回道:“珍大爷那边给了张家不知多少银子,那张家就不问了。”凤姐道:“这里头怎么又扯拉上什么张家李家咧呢?”兴儿回道:“奶奶不知道,这二奶奶……”刚说到这里,又自己打了个嘴巴,把凤姐儿倒怄笑了。两边的丫头也都抿嘴儿笑。兴儿想了想,说道:“那珍大奶奶的妹子……”凤姐儿接着道:“怎么样?快说呀。”兴儿道:“那珍大奶奶的妹子原来从小儿有人家的,姓张,叫什么张华,如今穷的待好讨饭。珍大爷许了他银子,他就退了亲了。”

  凤姐儿听到这里,点了点头儿,回头便望丫头们说道:“你们都听见了?小忘八崽子,头里他还说他不知道呢!”兴儿又回道:“后来二爷才叫人裱糊了房子,娶过来了。”凤姐道:“打那里娶过来的?”兴儿回道:“就在他老娘家抬过来的。”凤姐道:“好罢咧。”又问:“没人送亲么?”兴儿道:“就是蓉哥儿。还有几个丫头老婆子们,没别人。”凤姐道:“你大奶奶没来吗?”兴儿道:“过了两天,大奶奶才拿了些东西来瞧的。”凤姐儿笑了一笑,回头向平儿道:“怪道那两天二爷称赞大奶奶不离嘴呢。”掉过脸来又问兴儿,“谁伏侍呢?自然是你了。”兴儿赶着碰头不言语。凤姐又问,“前头那些日子说给那府里办事,想来办的就是这个了。”兴儿回道:“也有办事的时候,也有往新房子里去的时候。”凤姐又问道:“谁和他住着呢。”兴儿道:“他母亲和他妹子。昨儿他妹子各人抹了脖子了。”凤姐道:“这又为什么?”兴儿随将柳湘莲的事说了一遍。凤姐道:“这个人还算造化高,省了当那出名儿的忘八。”因又问道:“没了别的事了么?”兴儿道:“别的事奴才不知道。奴才刚才说的字字是实话,一字虚假,奶奶问出来只管打死奴才,奴才也无怨的。”

  凤姐低了一回头,便又指着兴儿说道:“你这个猴儿崽子就该打死。这有什么瞒着我的?你想着瞒了我,就在你那糊涂爷跟前讨了好儿了,你新奶奶好疼你。我不看你刚才还有点怕惧儿,不敢撒谎,我把你的腿不给你砸折了呢。”说着喝声“起去”。兴儿磕了个头,才爬起来,退到外间门口,不敢就走。凤姐道:“过来,我还有话呢。”兴儿赶忙垂手敬听。凤姐道:“你忙什么,新奶奶等着赏你什么呢?”兴儿也不敢抬头。凤姐道:“你从今日不许过去。我什么时候叫你,你什么时候到。迟一步儿,你试试!出去罢。”兴儿忙答应几个“是”,退出门来。凤姐又叫道:“兴儿!”兴儿赶忙答应回来。凤姐道:“快出去告诉你二爷去,是不是啊?”兴儿回道:“奴才不敢。”凤姐道:“你出去提一个字儿,堤防你的皮!”兴儿连忙答应着才出去了。凤姐又叫:“旺儿呢?”旺儿连忙答应着过来。凤姐把眼直瞪瞪的瞅了两三句话的工夫,才说道:“好旺儿,很好,去罢!外头有人提一个字儿,全在你身上。”旺儿答应着也出去了。

  凤姐便叫倒茶。小丫头子们会意,都出去了。这里凤姐才和平儿说:“你都听见了?这才好呢。”平儿也不敢答言,只好陪笑儿。凤姐越想越气,歪在枕上只是出神,忽然眉头一皱,计上心来,便叫:“平儿来。”平儿连忙答应过来。凤姐道:“我想这件事竟该这么着才好。也不必等你二爷回来再商量了。”未知凤姐如何办理,下回分解。

